

% This file is autogenerated. Don't edit it
\documentclass{article}

\usepackage{amsmath}              % need for subequations
\usepackage{amssymb}              % for symbols
\usepackage{hyperref}             % put links on stuff
\usepackage{cleveref}             % use for referencing figures/equations
                                  % hyperref must come before cleverref
\usepackage{natbib}               % biblio
\usepackage{color}                % use if color is used in text
\usepackage{soul}                 % hyphenation, strikethrough
\usepackage{graphicx}
\usepackage{parskip}

\bibliographystyle{unsrt}

\begin{document}

\title{A collection of papers}
\date{\today}

\author{gitbib}

\maketitle

\section{Section title tbd}
\subsection{Roads towards fault-tolerant universal quantum computation}

This paper can be cited with id \texttt{2017-fault-tolerant-computation}
\cite{2017-fault-tolerant-computation}.


Superconducting qubits [2013-superconducting-qubit-outlook=3].

Gottesman-Knill theorem says you need T in addition to S, H, and CNOT or
you get no quantum [1997-gottesman-thesis=9] (ref. 10).

Trivial codes can't have transversal implementations of all gates for
universal compuation (ref. 11) (ref. 12).

Surface code first as a topological memory [2002-surface-code=13]. Logical
qubit can be two holes in a code sheet (ref. 17) or two pairs of latice
defects or twists (ref. 18) (ref. 19).




\subsection{Quantum software}

This paper can be cited with id \texttt{2017-insight}
\cite{2017-insight}.


Nature ran a featurette where everyone mused about quantum computing and
how to make it useful. Includes [2017-fault-tolerant-computation],
[2017-quantum-programming-language],




\subsection{Programming languages and compiler design for realistic quantum hardware}

This paper can be cited with id \texttt{2017-quantum-programming-language}
\cite{2017-quantum-programming-language}.


Quantum/classical co-processor model described by [2015-quipper=19].




\subsection{First quantum computers need smart software}

This paper can be cited with id \texttt{2017-rigetti-quantum-software}
\cite{2017-rigetti-quantum-software}.


A comment that argues for good quantum software.




\subsection{Charge- and Flux-Insensitive Tunable Superconducting Qubit}

This paper can be cited with id \texttt{2017-tunable}
\cite{2017-tunable}.


Improve fluxonimum (ref. 10) (ref. 11) (ref. 12) (ref. 13) (ref. 14) (ref.
15) with "sweet spots". I think this is just simulations of how it would
behave w.r.t noise though.

Static qubit-qubit couplings with 2q-gates in hundreds of nanoseconds,
100us coherence, and fidelity of 99.1% (ref. 1) (ref. 2) (ref. 3).

Frequency tinable qubits: 20us coherence, 50ns 2q-gates and 99.44% fidelity
(ref. 4) (ref. 5). Fluctuations from flux noise ruin coherence (ref. 6)
(ref. 7) (ref. 8). Also not anharmonic enough means leaks to higher levels
(ref. 9).




\subsection{Demonstration of Universal Parametric Entangling Gates on a Multi-Qubit
  Lattice}

This paper can be cited with id \texttt{2017-multi-qubit}
\cite{2017-multi-qubit}.


Eight qubits in a ring, alternating fixed and tunable. Do 2q gates.




\subsection{Computational design of trimeric influenza-neutralizing proteins targeting the hemagglutinin receptor binding site}

This paper can be cited with id \texttt{2017-baker-antibody-design}
\cite{2017-baker-antibody-design}.


Uses computation to design an antibody for influenza A




\subsection{tICA-Metadynamics: Accelerating Metadynamics by Using Kinetically Selected Collective Variables}

This paper can be cited with id \texttt{2017-tica-metadynamics}
\cite{2017-tica-metadynamics}.


\subsection{Variational Koopman models: Slow collective variables and molecular kinetics from short off-equilibrium simulations}

This paper can be cited with id \texttt{2016-noe-reversible-tica}
\cite{2016-noe-reversible-tica}.


Provides a better way of "symetrizing" tICA correlation matrix. In tICA,
you assume that the dynamics are reversible. When we're learning from
finite data, this reversibility isn't respected. Historically, you take
your correlation matrix, add its transpose, and divide by two. This is an
especially poor approximation if you have many short trajectories. This
paper is analogous to the MLE method for symetrizing MSM counts matrices.




\subsection{Learning Important Features Through Propagating Activation Differences}

This paper can be cited with id \texttt{2017-deep-lift}
\cite{2017-deep-lift}.


Decompose ouput predictions




\subsection{Building a More Predictive Protein Force Field: A Systematic and Reproducible Route to AMBER-FB15}

This paper can be cited with id \texttt{2017-amber15fb}
\cite{2017-amber15fb}.


\subsection{Markov modeling reveals novel intracellular modulation of the human TREK-2 selectivity filter}

This paper can be cited with id \texttt{2016-trek2}
\cite{2016-trek2}.


\subsection{Cryo-EM Structure of the Open Human  Ether-à-go-go -Related K +  Channel hERG}

This paper can be cited with id \texttt{2017-herg-structure}
\cite{2017-herg-structure}.


\subsection{Set-free Markov state model building}

This paper can be cited with id \texttt{2017-set-free-msm}
\cite{2017-set-free-msm}.


Collection of m "base points" they make a gaussian RBF of distances to base
points. They normalize it to unity. This is the softmax function, but they
don't call it that.

They add base points adaptively and use PCCA+ to lump.

Note that they have stopped calling this "meshless" or "mesh-free",
probably because the regular MSM is also meshless. Now the abstract says
"This kind of meshless discretization..."




\subsection{Structures of closed and open states of a voltage-gated sodium channel}

This paper can be cited with id \texttt{2017-navab}
\cite{2017-navab}.


Open PD/VSD (5vb8) and closed PD/VSD/CTD (5vb2)




\subsection{The complete structure of an activated open sodium channel}

This paper can be cited with id \texttt{2017-open-full-navms}
\cite{2017-open-full-navms}.


First full-length, open conformation. [2017-navab] has a full-length
closed. This is NavMs and that is NavAb.




\subsection{Structure of a eukaryotic voltage-gated sodium channel at near-atomic resolution}

This paper can be cited with id \texttt{2017-euk-navpas}
\cite{2017-euk-navpas}.


First eukaryotic structure. CryoEM of cockroach NaV. Might not be a NaV. No
elecrophysiology can be performed. Was originally called PaFPC1, they
renamed it NaVPaS




\subsection{Simulating the Activation of Voltage Sensing Domain for a Voltage-Gated Sodium Channel Using Polarizable Force Field}

This paper can be cited with id \texttt{2017-vsd-only-pmf}
\cite{2017-vsd-only-pmf}.


They simulated only one domain. They took the NavAb vsd and connected it
via molecular dynamics to a homology model to a [sea
squirt](https://en.wikipedia.org/wiki/Ciona_intestinalis) vsd. They used a
polarizable force-field, which is more expensive than normal (fixed charge)
forcefields but that might be important for something so intertwined with
moving electrical charges!




\subsection{Identification of simple reaction coordinates from complex dynamics}

This paper can be cited with id \texttt{2016-sparsetica}
\cite{2016-sparsetica}.


The authors argue for a definition of the reaction coordinate as the
projection on the dominant eigenfunciton of the propogator. Notably, they
say that path-based coordinates are no good, because progress is only
defined along the path. They argue that the coordinate shouldn't depend on
start and end points. They say the projection should be maximally
predictive. This means finding the slowest modes. They note
[2006-nadler-diffusion-maps=61] and [2011-rohrdanz-diffusion-maps=62] have
used this definition.

They go on to show tICA finds this reaction coordinate. To make tICA more
interpretable, they develop an algorithm for introducing a sparsity
pattern. It's a pseudo-l0 regularization (made smooth so the optimization
works).

They also use a unique dihedral featurization: instead of taking the sine
and cosine to get around periodicity concerns; they project the values on a
bunch of evenly spaced von-mises (periodic gaussians) distributions around
the unit circle. Each dihedral is expanded into several numbers. It's like
a smooth histogramming. This probably won't work as the number of dihedrals
gets large (too many features).




\subsection{MSMBuilder: Statistical Models for Biomolecular Dynamics}

This paper can be cited with id \texttt{2016-msmbuilder3}
\cite{2016-msmbuilder3}.


\subsection{Optimized parameter selection reveals trends in Markov state models for protein folding}

This paper can be cited with id \texttt{2016-husic-msms}
\cite{2016-husic-msms}.


The authors perform GMRQ cross validation on the twelve
[2011-larsen-folding] folding trajectories to give guidelines for MSM
construction.

They present a flowchart for MSM construction that shows the three paths
towards clustering: from an rmsd distance metric, from features, or from
tICA learned on features.

They introduce GMRQ cross validation in the tradition of
[2015-mcgibbon-gmrq=44].

They present results but stress that you have to do your own cross
validataion to be sure. Some conclusions include: 1. tICA and PCA are
better than direct clustering of features 2. when using tica, you can use
kcenters, kmeans, or minibatch kmeans to the same effect

On one protein (2p6j) they look at all different features and show that
they vary a lot. It's unfortunate that this was only done on one protein.




\subsection{Commute Maps: Separating Slowly Mixing Molecular Configurations for Kinetic Modeling}

This paper can be cited with id \texttt{2016-commute-maps}
\cite{2016-commute-maps}.


Scale tIC coordinates by a function of the timescale. See also
[2015-kinetic-mapping].




\subsection{Hybrid computing using a neural network with dynamic external memory}

This paper can be cited with id \texttt{2016-neural-computers}
\cite{2016-neural-computers}.


Augment deep networks with an external memory (RAM) matrix.

Bart says: "TL;DR: This work follows a line of research that teaches
deep-nets to learn algorithmic tasks (addition, sorting, multiplication,
key-value look-up). This paper goes a bit further and teaches their network
to do shortest-path finding in graphs and demonstrates on maps of the
London underground. Cool demo with nice results, but the hype-machine has
blown it out of proportion (check out the FT article for a breathless take
claiming thinking computers are one step closer...)"




\subsection{Automatic chemical design using a data-driven continuous representation
  of molecules}

This paper can be cited with id \texttt{2016-aspuru-mol-feat}
\cite{2016-aspuru-mol-feat}.


The authors train an auto-encoder to provide a vector representation for
small molecules. Small molecules are graphs with varying sizes, so they're
hard to feed into neural nets (which require fixed-length bitvectors). By
fusing together an encoder and decoder (and making the "middle"
representation sufficiently small), they learn a vector representation.

The authors lean heavily on [arxiv:1511.06349=25] to autoencode SMILES
strings.

They use a variational autoencoder (noisy) to avoid "dead zones" in latent
space.

They optomize OLED properties as an example.




\subsection{Modelling proteins’ hidden conformations to predict antibiotic resistance}

This paper can be cited with id \texttt{2016-msm-cryptic-binding}
\cite{2016-msm-cryptic-binding}.


Labmate summarizes:

They generated ensembles using MD, then docked to those ensembles, then
re-weighted the docking scores based on the MSM. This gave a huge
improvement in the predictive power of docking to predict affinity/potency.
It turned an inverse relationship (when docking using xtal structures) into
a highly correlated trend.

They confirmed their hypothesis about the protein flexibility by using a
mass spec. method.

They identified a loop movement important in the anti-antibacterial
activity of the enzyme that was different from one previously
proposed/suspected.

They proposed mutants that would stabilize their proposed loop, and tested
them experimentally.

The power of using the MSM to re-weight other analyses is also very
encouraging to see yet again. Also note that they did all this with what
looks like a pretty low amount of aggregate sampling (few microseconds per
mutant).




\subsection{A functional architecture for scalable quantum computing}

This paper can be cited with id \texttt{2016-scalable}
\cite{2016-scalable}.


Quantum simulation algorithsm (ref. 1) (ref. 2) (ref. 3).

Quantum machine learning (ref. 4)

Quantum error correction benchmarks (ref. 5) (ref. 6) (ref. 7).

Variational quantum eigensolvers (ref. 8) (ref. 9) (ref. 10).

Correlated material simulations (ref. 11).

Approximate optimization (ref. 12).

For the problems of catalysts (ref. 13) and high temperature
superconductivity (ref. 9) show promise.

Cryo operation and superconducting materials means no sissipation
preserving quantum coherance.

Transmon qubits have large coherence time (ref. 14). Fluxonium qubits have
wide frequency tunability and strong nonlinearity (ref. 15). This means
fluxonium are better for two-qubit gates.

Quantum limited amplifiers (ref. 16) (ref. 17) (ref. 18): Josephson
parametric amplifier, Josephson bifurcation amplifier, and Josephson
parametric converter. Non-linear resonators.

Can do rotations Rx and Ry on any qubit. Can do SWAP between any transmon
and fluxonium. Can do CPhase between any fluxonium and half the transmons.
All gates can be made with these primitives (ref. 19).

Introduce "TQF" estimate of width * depth of quantum circuit you can run.
(ref. 1) runs electronic structure for very small molecules.

Transmon can be "data" for surface code error correction (ref. 24) (ref.
25) and fluxonium as ancillas for parity measurement.




\subsection{Quantum-Enhanced Machine Learning}

This paper can be cited with id \texttt{2016-quantum-ml}
\cite{2016-quantum-ml}.


\subsection{Structural analysis of high-dimensional basins of attraction}

This paper can be cited with id \texttt{2016-mbar-volumes}
\cite{2016-mbar-volumes}.


Use multistate benett acceptance (MBAR) to find volumes in high dimensions.




\subsection{Osprey: Hyperparameter Optimization for Machine Learning}

This paper can be cited with id \texttt{2016-osprey}
\cite{2016-osprey}.


\subsection{Neural Coarse-Graining: Extracting slowly-varying latent degrees of
  freedom with neural networks}

This paper can be cited with id \texttt{2016-guttenberg-deep-slow}
\cite{2016-guttenberg-deep-slow}.


Somehow uses deep networks to extract slow modes from dynamical signals.




\subsection{Instantaneous ion configurations in the K+ion channel selectivity filter revealed by 2D IR spectroscopy}

This paper can be cited with id \texttt{2016-kratochvil-soft-knock}
\cite{2016-kratochvil-soft-knock}.


\subsection{Structure of the voltage-gated calcium channel Cav1.1 at 3.6 Å resolution}

This paper can be cited with id \texttt{2016-cav-structure}
\cite{2016-cav-structure}.


Cav structure. What's the diference between [2015-cav-structure]


\subsection{A Practical Quantum Instruction Set Architecture}

This paper can be cited with id \texttt{2016-quil}
\cite{2016-quil}.


\subsection{The Receptor Site and Mechanism of Action of Sodium Channel Blocker Insecticides}

This paper can be cited with id \texttt{2016-btx-insect}
\cite{2016-btx-insect}.


Homology model to open cockroach channel and docking study




\subsection{Scalable Quantum Simulation of Molecular Energies}

This paper can be cited with id \texttt{2016-h2-vqe}
\cite{2016-h2-vqe}.


Solves molecular hydrogen with variational quantum eigensolver (which is
hybrid quantum - classical) and compares to trotterization and quantum
phase estimation. The VQE is better.




\subsection{Ensembler: Enabling High-Throughput Molecular Simulations at the Superfamily Scale}

This paper can be cited with id \texttt{2015-ensembler}
\cite{2015-ensembler}.


Automatic generation of homology models of protein families




\subsection{Polymodal activation of the TREK-2 K2P channel produces structurally distinct open states}

This paper can be cited with id \texttt{mcclenaghan2016polymodal}
\cite{mcclenaghan2016polymodal}.


\subsection{TensorFlow: A system for large-scale machine learning}

This paper can be cited with id \texttt{2016-tensorflow}
\cite{2016-tensorflow}.


\subsection{Transmembrane Potential Modeling: Comparison between Methods of Constant Electric Field and Ion Imbalance}

This paper can be cited with id \texttt{2016-melcr-membrane-compare}
\cite{2016-melcr-membrane-compare}.


Compare external electric field with ion imbalance.




\subsection{HTMD: High-Throughput Molecular Dynamics for Molecular Discovery}

This paper can be cited with id \texttt{2016-htmd}
\cite{2016-htmd}.


This can make MSMs in addition to being a one-stop shop for running MD.

It's available under an academic license.




\subsection{Markov State Models and tICA Reveal a Nonnative Folding Nucleus in Simulations of NuG2}

This paper can be cited with id \texttt{2016-schwantes-nug2}
\cite{2016-schwantes-nug2}.


They find an intermediate in [2011-larsen-folding] NuG2 trajectories that
is a register shift that was missed before tICA+MSM.




\subsection{Statistical models of protein conformational dynamics}

This paper can be cited with id \texttt{2016-mcgibbon-thesis}
\cite{2016-mcgibbon-thesis}.


Chapter 1 is a bespoke introduction to MD and MSMs

Chapter 2 is adapted from [2013-mcgibbon-kdml=37].

Chapter 3 is adapted from [2014-mcgibbon-hmm=92].

Chapter 4 is adapted from [2015-ratematrix=120].

Chapter 5 is adapted from [2014-mcgibbon-bic=162].

Chapter 6 is adapted from [2015-mcgibbon-gmrq=214].

Chapter 7 is adapted from [2016-sparsetica].

Chapter 8 is adapted from [2015-mdtraj].




\subsection{Allosteric coupling between proximal C-terminus and selectivity filter is facilitated by the movement of transmembrane segment 4 in TREK-2 channel}

This paper can be cited with id \texttt{ren2016allosteric}
\cite{ren2016allosteric}.


\subsection{Notes on the Theory of Markov Chains in a Continuous State Space}

This paper can be cited with id \texttt{2016-mcgibbon-notes}
\cite{2016-mcgibbon-notes}.


\subsection{A Non-canonical Voltage-Sensing Mechanism Controls Gating in K2P K+ Channels}

This paper can be cited with id \texttt{schewe2016non}
\cite{schewe2016non}.


\subsection{Convergent Substitutions in a Sodium Channel Suggest Multiple Origins of Toxin Resistance in Poison Frogs}

This paper can be cited with id \texttt{2016-btx-frogs}
\cite{2016-btx-frogs}.


Evolutionary analysis and docking study on NaV 1.4 in frogs immune to
toxins




\subsection{CHARMM-GUI Input Generator for NAMD, GROMACS, AMBER, OpenMM, and CHARMM/OpenMM Simulations Using the CHARMM36 Additive Force Field}

This paper can be cited with id \texttt{2016-charmm-gui}
\cite{2016-charmm-gui}.


\subsection{Structure of the voltage-gated calcium channel Cav1.1 complex}

This paper can be cited with id \texttt{2015-cav-structure}
\cite{2015-cav-structure}.


also a cav structure.


\subsection{Generating Sentences from a Continuous Space}

This paper can be cited with id \texttt{arxiv:1511.06349}
\cite{arxiv:1511.06349}.


Advances in autoencoding text, used by [2016-aspuru-mol-feat].


\subsection{PyEMMA 2: A Software Package for Estimation, Validation, and Analysis of Markov Models}

This paper can be cited with id \texttt{2015-pyemma}
\cite{2015-pyemma}.


\subsection{Estimation and uncertainty of reversible Markov models}

This paper can be cited with id \texttt{2015-uncertainty-estimation}
\cite{2015-uncertainty-estimation}.


Reversible estimates for MSMs


\subsection{MDTraj: A Modern Open Library for the Analysis of Molecular Dynamics Trajectories}

This paper can be cited with id \texttt{2015-mdtraj}
\cite{2015-mdtraj}.


\subsection{A Basis Set for Peptides for the Variational Approach to Conformational Kinetics}

This paper can be cited with id \texttt{2015-amino-acid-basis}
\cite{2015-amino-acid-basis}.


Authors simulate individual (capped) amino acids for 1us / each and
construct (mini-)MSMs on each one. They use the outerproduct of these
mini-MSMs to serve as a basis set for peptides. MiniMSMs are on a grid in
phi-psi angles. Since each miniMSM has approx 3 modes, the full basis would
be 3^(N), which is way too big! They call the second and third modes
"excited states" and use a basis set that contains a singly-exited residue.
E.g. 11111 + [ [21111, 121111, 112111, 111211, ...] ].

Alanine preceded by a proline is taken as a special case.




\subsection{GROMACS: High performance molecular simulations through multi-level parallelism from laptops to supercomputers}

This paper can be cited with id \texttt{2015-gromacs}
\cite{2015-gromacs}.


\subsection{Programming the quantum future}

This paper can be cited with id \texttt{2015-quipper}
\cite{2015-quipper}.


Quantum programming language implemented inside Haskell. Invisions quantum
co-processor.




\subsection{Efficient maximum likelihood parameterization of continuous-time Markov processes}

This paper can be cited with id \texttt{2015-ratematrix}
\cite{2015-ratematrix}.


\subsection{A critical appraisal of Markov state models}

This paper can be cited with id \texttt{2015-schutte-msm}
\cite{2015-schutte-msm}.


Transfer operator [1999-schutte-msm=1].

Coarse grain MSM states [2000-pcca=2] [2005-pcca=3].

Meshless MSMs [2006-meshless-msm-thesis=24] [2011-meshless-msm=32]
[2011-meshless-msm=33]. Wikipedia says these are also called "meshfree"
methods.




\subsection{Kinetic distance and kinetic maps from molecular dynamics simulation}

This paper can be cited with id \texttt{2015-kinetic-mapping}
\cite{2015-kinetic-mapping}.


Scale tIC coorindates by the eigenvalue. See also [2016-commute-maps].




\subsection{Automated construction of order parameters for analyzing simulations of protein folding and water dynamics}

This paper can be cited with id \texttt{2015-schwantes-thesis}
\cite{2015-schwantes-thesis}.


Section 1.2 is adapted from [2015-schwantes-ktica=27] and
[2014-mcgibbon-bic=28].

Chapter 2 is adapted from [2014-mcgibbon-bic=28].

Chapter 3 is adapted from [2013-schwantes-tica=73].

Chapter 4 is adapted from [2016-schwantes-nug2=122].

Chapter 5 is adapted from [2015-schwantes-ktica=27].

Chapter 6 is supposed to have been submitted for publication.




\subsection{Variational cross-validation of slow dynamical modes in molecular kinetics}

This paper can be cited with id \texttt{2015-mcgibbon-gmrq}
\cite{2015-mcgibbon-gmrq}.


\subsection{K2P channel gating mechanisms revealed by structures of TREK-2 and a complex with Prozac}

This paper can be cited with id \texttt{2015-carpenter-structures}
\cite{2015-carpenter-structures}.


Structures of up and down trek2.

Cites [2010-k2p-review=1] for background.


\subsection{Conserve Water: A Method for the Analysis of Solvent in Molecular Dynamics}

This paper can be cited with id \texttt{2015-wetmsm}
\cite{2015-wetmsm}.


Solvent-shells featurization for including solvent in MSM construction.




\subsection{Gaussian Markov transition models of molecular kinetics}

This paper can be cited with id \texttt{2015-gaussian-msms}
\cite{2015-gaussian-msms}.


Variational method [2013-noe-variational=26] [2014-nuske-variational=27].

MSM is variational with step functions [2013-noe-variational=26].

"Markov transition models (MTMs)", specifically Gaussian mixtures (GMTM).




\subsection{Markov State Models Provide Insights into Dynamic Modulation of Protein Function}

This paper can be cited with id \texttt{2015-shukla-msm-review}
\cite{2015-shukla-msm-review}.


\subsection{Modeling Molecular Kinetics with tICA and the Kernel Trick}

This paper can be cited with id \texttt{2015-schwantes-ktica}
\cite{2015-schwantes-ktica}.


They introduce kernel tICA as an extension to tICA. This is useful to get
non-linear solutions to the tICA equation. They claim you can estimate
eigenprocesses without building an MSM.

They briefly introduce the transfer operator. They introduce the
variational principle of conformation dynamics per [2011-prinz=25]. They
introduce tICA as maximizing the autocorrelation. They say that solutions
to tICA are the same as solutions to the variational problem per
[2013-noe-tica=28]. Linearity makes them crude solutions.

They explain that a natural approach to introduce non-linearity is to
expand the original representation into a higher dimensional space and do
tICA there. They say this is impractical. The expanded space probably has
to be huge. You can perform analysis in the big representation without
explicitly representing it by using the "kernel trick". They reproduce an
example of the kernel trick from [1998-scholkopf-kernel-pca=39].

They re-write the tICA problem only in terms of inner products so you can
apply the kernel trick. They introduce normalization. They choose a
gaussian kernel. They simulate a four-well potential, muller potential,
alanine dipeptide, and fip35ww. They need to do MLE cross validation over
parameters (kernel width and regularization strength).

This uses so much RAM! Huge matrices to solve (that scale with the amount
of data!!)




\subsection{State-independent intracellular access of quaternary ammonium blockers to the pore of TREK-1}

This paper can be cited with id \texttt{rapedius2012state}
\cite{rapedius2012state}.


\subsection{Molecular regulations governing TREK and TRAAK channel functions}

This paper can be cited with id \texttt{noel2011molecular}
\cite{noel2011molecular}.


\subsection{Potassium ions line up}

This paper can be cited with id \texttt{hummer2014potassium}
\cite{hummer2014potassium}.


\subsection{Ion permeation in K+ channels occurs by direct Coulomb knock-on}

This paper can be cited with id \texttt{2014-kopfer-hard-knock}
\cite{2014-kopfer-hard-knock}.


Introduces a new way of simulating a membrane potential: They stack two
membranes on top of one another, creating an "inside" between the two. This
doesn't hurt simulation throughput, because you get twice as much protein
motion data in the same amount of simulation time (ignore extra factor of
log n in system size). This seems to be a refinement on their earlier work
in [2011-kutzner-double-membrane=25].

They hide the startling fact that every time an ion moves through the
channel, they have to instantaneously move it back inside. Benoit has
argued that this instantaneous jump, which can be a 100 mV difference is
rediculous.

The main point of this paper is that ions translocate through the four
sites of potassium channel without any waters between them. This "hard
knock" mechanism is in contrast to a "soft knock" mechanism where the
ion-ion interactions are softened by interviening waters.

They re-refine the xray data to show it is consistent with the hard-knock
mechanism.




\subsection{Perspective: Markov models for long-timescale biomolecular dynamics}

This paper can be cited with id \texttt{2014-msm-perspective}
\cite{2014-msm-perspective}.


Very good perspective on the importance of analysis (particularly MSM
analysis) for understanding large, modern MD datasets. Money quote: "we
believe that quantitative analysis has increasingly become a limiting
factor in the application of MD"


\subsection{A hydrophobic barrier deep within the inner pore of the TWIK-1 K2P potassium channel}

This paper can be cited with id \texttt{aryal2014hydrophobic}
\cite{aryal2014hydrophobic}.


\subsection{Statistical Model Selection for Markov Models of Biomolecular Dynamics}

This paper can be cited with id \texttt{2014-mcgibbon-bic}
\cite{2014-mcgibbon-bic}.


This is before [2015-mcgibbon-gmrq] GRMQ cross-validation. They explicitly
find the volume of voronoi cells (in low number of tIC space) to find a
likelihood. They use AIC/BIC to find the number of states to use. Finding
volumes is tough and you still can't compare across protocols (so you can
basically only scan number of states or clustering method), but! this was
the first paper to seriously suggest using a smaller number of states to
avoid overfitting.




\subsection{Derivation of coarse-grained potentials via multistate iterative Boltzmann inversion}

This paper can be cited with id \texttt{2014-moore-coarsegrain}
\cite{2014-moore-coarsegrain}.


Coarse-graining?


\subsection{Identifying and attacking the saddle point problem in high-dimensional
  non-convex optimization}

This paper can be cited with id \texttt{2014-ganguli-saddle-points}
\cite{2014-ganguli-saddle-points}.


Labmate summarizes:

This one is a really cool paper.  One of those "we've all been doing it
wrong" papers that could have a big impact.  Their main conclusions are

1. When optimizing functions in high dimensional spaces, saddle points are
a much bigger problem than local minima.  There are far more of them, and
the few local minima that do exist mostly have values only slightly worse
than the global minimum.

2. Standard optimization methods deal really badly with saddle points (and
hence work really badly in high dimensional spaces).  First order methods
like gradient descent start taking tiny steps, so they take a really long
time to escape.  Quasi-Newton methods are even worse.  They just converge
to the saddle point and never escape.

3. They describe a new approach that doesn't have these problems and goes
right through saddle points without slowing down.

They do all this in the context of neural networks, but it likely applies
just as well to other high dimensional optimization problems.  Proteins,
for example.  When you use an algorithm like L-BFGS for energy
minimization, it's probably converging to a saddle point, not a local
minimum.  It could be really interesting to try their method.  Could we
fold a protein to the native state just by a straightforward energy
minimization?

Force field optimization is another case whether this approach could be
really useful.

They also show that at a saddle point, there's a strong monotonic
relationship between the error and the fraction of negative eigenvalues of
the Hessian.  Potentially that could be used as a way to measure how far
you are from the global minimum.  For example, when optimizing force field
parameters, it would tell you whether your parameters are close to optimal,
or whether there's still a lot of room to improve them further.




\subsection{Using Markov state models to study self-assembly}

This paper can be cited with id \texttt{perkett{\_}using{\_}2014}
\cite{perkett_using_2014}.


Cited by [2015-wetmsm] as successful MSM application for self-assembly




\subsection{Building Force Fields: An Automatic, Systematic, and Reproducible Approach}

This paper can be cited with id \texttt{2014-forcebalance}
\cite{2014-forcebalance}.


\subsection{Fs MD Trajectories}

This paper can be cited with id \texttt{2014-fs-peptide}
\cite{2014-fs-peptide}.


\subsection{Do Lipids Show State-dependent Affinity to the Voltage-gated Potassium Channel KvAP?}

This paper can be cited with id \texttt{faure{\_}lipids{\_}2014}
\cite{faure_lipids_2014}.


Cited by [2015-wetmsm] where lipids are important for modulation of ion
channel function




\subsection{Variational Approach to Molecular Kinetics}

This paper can be cited with id \texttt{2014-nuske-variational}
\cite{2014-nuske-variational}.


This paper is largely redundant with [2013-noe-variational=65]. They cite
it as such: "Following the recently introduced variational principle for
metastable stochastic processes,(65) we propose a variational approach to
molecular kinetics."

They perform their variational approach on 2- and 10-alanine in addition to
1D potentials.

This comes after tICA and cites [2013-schwantes-tica=57] and
[2013-noe-tica=58] in the intro, but does nothing further with it. In
particular, they don't note that tICA is just another choice of basis set.

They cite their error paper [2010-msm-error=55].




\subsection{Markov state models of biomolecular conformational dynamics}

This paper can be cited with id \texttt{2014-chodera-msm}
\cite{2014-chodera-msm}.


Overview of MSMs, stressing eigensystem and variational approach. Includes
further reading suggestions.


\subsection{Activation pathway of Src kinase reveals intermediate states as targets for drug design}

This paper can be cited with id \texttt{2014-shukla-src-kinase}
\cite{2014-shukla-src-kinase}.


MSM analysis of c-Src kinase. The MSMBuilder paper uses the dataset from
this paper as an example.


\subsection{A Molecular Interpretation of 2D IR Protein Folding Experiments with Markov State Models}

This paper can be cited with id \texttt{baiz{\_}molecular{\_}2014}
\cite{baiz_molecular_2014}.


Cited by [2015-wetmsm] as successful MSM application for folding




\subsection{Spectral Rate Theory for Two-State Kinetics}

This paper can be cited with id \texttt{2014-prinz-rate}
\cite{2014-prinz-rate}.


\subsection{An Antifreeze Protein Folds with an Interior Network of More Than 400 Semi-Clathrate Waters}

This paper can be cited with id \texttt{sun{\_}antifreeze{\_}2014}
\cite{sun_antifreeze_2014}.


Cited by [2015-wetmsm] where solvent is important




\subsection{Lipid14: The Amber Lipid Force Field}

This paper can be cited with id \texttt{2014-lipid14}
\cite{2014-lipid14}.


\subsection{Best Practices for Scientific Computing}

This paper can be cited with id \texttt{2014-scicomp-best-practices}
\cite{2014-scicomp-best-practices}.


\subsection{Structure of a Prokaryotic Sodium Channel Pore Reveals Essential Gating Elements and an Outer Ion Binding Site Common to Eukaryotic Channels}

This paper can be cited with id \texttt{2014-closed-navae1p}
\cite{2014-closed-navae1p}.


Closed Pore and CTD in pore-only NavAe. Correlates CTD neck unfolding with
activation.




\subsection{An exploratory study of the pull-based software development model}

This paper can be cited with id \texttt{2014-pull-requests}
\cite{2014-pull-requests}.


Pull-request based development model




\subsection{Cloud-based simulations on Google Exacycle reveal ligand modulation of GPCR activation pathways}

This paper can be cited with id \texttt{2014-kohlhoff-exacycle}
\cite{2014-kohlhoff-exacycle}.


They used Google's Exacycle to do these simulations. You can cite this for
more examples of distributed computing. It's ostensibly about GPCRs.


\subsection{Projected and hidden Markov models for calculating kinetics and metastable states of complex molecules}

This paper can be cited with id \texttt{2013-noe-hmm}
\cite{2013-noe-hmm}.


\subsection{Rapid Exploration of Configuration Space with Diffusion-Map-Directed Molecular Dynamics}

This paper can be cited with id \texttt{2013-diffusion-map-sampling}
\cite{2013-diffusion-map-sampling}.


Use diffusion maps to run umberlla sampling




\subsection{Role of the C-terminal domain in the structure and function of tetrameric sodium channels}

This paper can be cited with id \texttt{2013-open-pore-navms}
\cite{2013-open-pore-navms}.


Pore only open conformation. Supposed to have the CTD but rcsb pdb doesn't
show it. Proposes CTD role in gating.




\subsection{Inferring protein structure and dynamics from simulation and experiment}

This paper can be cited with id \texttt{2013-beauchamp-thesis}
\cite{2013-beauchamp-thesis}.


\subsection{Systematic Improvement of a Classical Molecular Model of Water}

This paper can be cited with id \texttt{2013-tip3p-fb}
\cite{2013-tip3p-fb}.


Cited in [2015-wetmsm] intro as example of better water models. Optimizes
tip3p and tip4p parameters.




\subsection{Learning Kinetic Distance Metrics for Markov State Models of Protein Conformational Dynamics}

This paper can be cited with id \texttt{2013-mcgibbon-kdml}
\cite{2013-mcgibbon-kdml}.


Learn scaling of coordinates to better approximate kinetics? Redundant with
tICA.




\subsection{Identification of slow molecular order parameters for Markov model construction}

This paper can be cited with id \texttt{2013-noe-tica}
\cite{2013-noe-tica}.


The Noe group introduces tica concomitantly with [2013-schwantes-tica].
They use the variational approach from [2013-noe-variational] to derive the
tICA equation. They cite a 2001 book about independent component analysis.


\subsection{Improvements in Markov State Model Construction Reveal Many Non-Native Interactions in the Folding of NTL9}

This paper can be cited with id \texttt{2013-schwantes-tica}
\cite{2013-schwantes-tica}.


The Pande group introduces tica concomitantly with [2013-noe-tica]. This
paper uses PCA as inspiration and cites signal processing literature.


\subsection{Superconducting Circuits for Quantum Information: An Outlook}

This paper can be cited with id \texttt{2013-superconducting-qubit-outlook}
\cite{2013-superconducting-qubit-outlook}.


\subsection{To milliseconds and beyond: challenges in the simulation of protein folding}

This paper can be cited with id \texttt{2013-milliseconds-folding}
\cite{2013-milliseconds-folding}.


The state of folding simulations as it was in 2013. Has a nice plot of
folding time by year by lab. Discusses the state of MSMs for analysis.
Maybe cite this if you're doing folding or want to talk about how
timescales are getting longer (and analysis is getting harder). The
references include "recommended readings", which is nice.


\subsection{OpenMM 4: A Reusable, Extensible, Hardware Independent Library for High Performance Molecular Simulation}

This paper can be cited with id \texttt{2013-openmm}
\cite{2013-openmm}.


\subsection{Building Markov state models with solvent dynamics}

This paper can be cited with id \texttt{gu{\_}building{\_}2013}
\cite{gu_building_2013}.


[2015-wetmsm] method based off of this.




\subsection{A Variational Approach to Modeling Slow Processes in Stochastic Dynamical Systems}

This paper can be cited with id \texttt{2013-noe-variational}
\cite{2013-noe-variational}.


I think the point of this versus [2014-nuske-variational] is to be "protein
agnostic". They allude to proteins, but say this is more general. Their
example is a double-well potential.

They introduce the propogator formalism and stipulate that dynamics can be
seperated into "fast" and "slow" components. In contrast to a quantum
mechanics Hamiltonian, we don't know the propogator here. You have to infer
it from data.

They claim the error bound derived in [2010-msm-error=34] is not
constructive, whereas this method *is* constructive.

Math section heavily cites [2010-msm-error=34].

They adapt the Rayleigh variational principle from quantum mechanics, and
cite [1989-szabo-ostlund-qm=43]. They show that the autocorrelation of the
true first dynamical eigenfunction is its eigenvalue, and an estimate of
the first dynamical eigenfunction necessarily has a smaller eigenvalue.
This sets the variational bound. In terms of names that don't seem to be
used now that we're in the future: the Ritz method is for when you have no
overlap integrals (e.g. MSMs) and the Roothan-Hall method is for when you
do (tICA).

They put it to the test on a double well potential. They use indicator
basis functions to make an MSM; hermite basis functions so they still have
no overlap integrals, but smooth functions; and gaussian basis functions
(with overlap integrals). This must have come before tICA because there is
no mention made of it, even though it would fit in nicely.




\subsection{Kinetic characterization of the critical step in HIV-1 protease maturation}

This paper can be cited with id \texttt{sadiq{\_}kinetic{\_}2012}
\cite{sadiq_kinetic_2012}.


Cited by [2015-wetmsm] as successful MSM application for kinase activation




\subsection{Structure of a bacterial voltage-gated sodium channel pore reveals mechanisms of opening and closing}

This paper can be cited with id \texttt{2012-partially-open-pore-navms}
\cite{2012-partially-open-pore-navms}.


Pore only partially open state. This may or may not collapse to the closed
state in molecular dynamics.




\subsection{A Meshless Discretization Method for Markov State Models Applied to Explicit Water Peptide Folding Simulations}

This paper can be cited with id \texttt{2013-meshless-msm}
\cite{2013-meshless-msm}.


Soften the hard clustering [2006-meshless-msm-thesis=37].

Cite Shepard's approach [1968-shepard-method=30] like
[2006-meshless-msm-thesis] does to introduce the softmax function as basis
functions with softness parameter alpha. Note that this is not Shepard's
method.

They frame everything in the context of lumping and PCCA+ and use
ZIBgridfree to simulate trialanine faster than unbiased (100ns vs 10ns).




\subsection{Distribution of Reciprocal of Interatomic Distances: A Fast Structural Metric}

This paper can be cited with id \texttt{2012-drid}
\cite{2012-drid}.


A unique featurization that encodes each atom by the first ~3 moments of
its distribution of 1/distance to every other atom. Cite this if you use
this featurization.


\subsection{Slow Unfolded-State Structuring in Acyl-CoA Binding Protein Folding Revealed by Simulation and Experiment}

This paper can be cited with id \texttt{voelz{\_}slow{\_}2012}
\cite{voelz_slow_2012}.


Cited by [2015-wetmsm] as successful MSM application for folding




\subsection{EMMA: A Software Package for Markov Model Building and Analysis}

This paper can be cited with id \texttt{2012-jemma}
\cite{2012-jemma}.


The previous, java version of EMMA. Look at [2015-pyemma] instead.




\subsection{Crystal structure of an orthologue of the NaChBac voltage-gated sodium channel}

This paper can be cited with id \texttt{2012-closed-navrh}
\cite{2012-closed-navrh}.


Pore and Voltage Sensing domains in a closed state.




\subsection{Crystal structure of a voltage-gated sodium channel in two potentially inactivated states}

This paper can be cited with id \texttt{2012-closed-asym-navab}
\cite{2012-closed-asym-navab}.


Pore and Voltage Sensing domains in an asymmetric closed state. NavAb.




\subsection{Routine Microsecond Molecular Dynamics Simulations with AMBER on GPUs. 1. Generalized Born}

This paper can be cited with id \texttt{2012-amber}
\cite{2012-amber}.


\subsection{Are Protein Force Fields Getting Better? A Systematic Benchmark on 524 Diverse NMR Measurements}

This paper can be cited with id \texttt{1999-ff99}
\cite{1999-ff99}.


\subsection{Constant electric field simulations of the membrane potential illustrated with simple systems}

This paper can be cited with id \texttt{2012-roux-efield}
\cite{2012-roux-efield}.


Constant electric field




\subsection{Nyström method vs random fourier features: A theoretical and empirical comparison}

This paper can be cited with id \texttt{2012-nystroem}
\cite{2012-nystroem}.


\subsection{Estimating the Eigenvalue Error of Markov State Models}

This paper can be cited with id \texttt{2012-eigenvalue-error}
\cite{2012-eigenvalue-error}.


\subsection{Structural basis for gating charge movement in the voltage sensor of a sodium channel}

This paper can be cited with id \texttt{2012-vsd-mechanism}
\cite{2012-vsd-mechanism}.


Homology modelling to resting, intermediate and activated states of voltage
sensing domains. Disulfide locking experiments. Cites
[doi:10.1073/pnas.0806486105] and [doi:10.1073/pnas.0912307106] and
[doi:10.1073/pnas.1116449108]




\subsection{Markov State Model Reveals Folding and Functional Dynamics in Ultra-Long MD Trajectories}

This paper can be cited with id \texttt{lane{\_}markov{\_}2011}
\cite{lane_markov_2011}.


Cited by [2015-wetmsm] as successful MSM application for folding




\subsection{How Fast-Folding Proteins Fold}

This paper can be cited with id \texttt{2011-larsen-folding}
\cite{2011-larsen-folding}.


The authors simulated folding trajectories for 12 small proteins. The
simulations were between 100 us and 1 ms. This paper was a considerable
advance for the field, and more or less closed the book on molecular
dynamics for folding.




\subsection{MSMBuilder2: Modeling Conformational Dynamics on the Picosecond to Millisecond Scale}

This paper can be cited with id \texttt{2011-msmbuilder2}
\cite{2011-msmbuilder2}.


\subsection{The pore structure and gating mechanism of K2P channels}

This paper can be cited with id \texttt{piechotta2011pore}
\cite{piechotta2011pore}.


\subsection{Computational Electrophysiology: The Molecular Dynamics of Ion Channel Permeation and Selectivity in Atomistic Detail}

This paper can be cited with id \texttt{2011-kutzner-double-membrane}
\cite{2011-kutzner-double-membrane}.


They introduce the double-membrane scheme for measuring ion conductance.
They do it on a big ol' beta barrel.




\subsection{Multiple modalities converge on a common gate to control K2Pchannel function}

This paper can be cited with id \texttt{bagriantsev2011multiple}
\cite{bagriantsev2011multiple}.


\subsection{The crystal structure of a voltage-gated sodium channel}

This paper can be cited with id \texttt{2011-closed-navab}
\cite{2011-closed-navab}.


Pore and Voltage Sensing domains in a closed state. First NaV Structure.
Prokaryote NavAb




\subsection{Soft Versus Hard Metastable Conformations in Molecular Simulations}

This paper can be cited with id \texttt{2011-meshless-msm}
\cite{2011-meshless-msm}.


They note MSM is a meshfree method with characteristic basis functions.

They define a "hard decomposition" in the obvious way. They define a "soft
decomposition" also as a partitioning of unity, but allowing overlap.

They still do PCCA+ and it's unclear what shape function they're using for
softness. As an example, they lump 504 soft states into 5 macrostates of
alanine dipeptide, sometimes spelled alanin dipeptid




\subsection{Complete reconstruction of an enzyme-inhibitor binding process by molecular dynamics simulations}

This paper can be cited with id \texttt{buch{\_}complete{\_}2011}
\cite{buch_complete_2011}.


Cited by [2015-wetmsm] as successful MSM application for protein-ligand
binding

Cited by [2015-wetmsm] where solvent is treated by grid of voxels.




\subsection{Markov models of molecular kinetics: Generation and validation}

This paper can be cited with id \texttt{2011-prinz}
\cite{2011-prinz}.


Fantastic in-depth intro to MSMs. Figure 1 in this paper is necessary for
understanding eigenvectors. This defines and relates the propogator and
transfer operator. This shows how we compute timescales from eigenvectors.
This discusess state decomposition error and shows that many states are
needed in transition regions.

quote: it is clear that a “sufficiently fine” partitioning will be able to
resolve “sufficient” detail [2010-msm-error].

Cites [2004-nina-msm] for use of the term "MSM".




\subsection{MDAnalysis: A toolkit for the analysis of molecular dynamics simulations}

This paper can be cited with id \texttt{2011-mdanalysis}
\cite{2011-mdanalysis}.


\subsection{Determination of reaction coordinates via locally scaled diffusion map}

This paper can be cited with id \texttt{2011-rohrdanz-diffusion-maps}
\cite{2011-rohrdanz-diffusion-maps}.


\subsection{Slow dynamics in protein fluctuations revealed by time-structure based independent component analysis: The case of domain motions}

This paper can be cited with id \texttt{2011-japan-tica}
\cite{2011-japan-tica}.


Probably the first application of tICA to MD.


\subsection{Protein kinases: evolution of dynamic regulatory proteins}

This paper can be cited with id \texttt{2011-kinase-review}
\cite{2011-kinase-review}.


Review of protein kinases. The MSMBuilder paper uses a kinase MD dataset as
an example.


\subsection{Scikit-learn: Machine Learning in Python}

This paper can be cited with id \texttt{2011-sklearn}
\cite{2011-sklearn}.


\subsection{Structural Inhomogeneity of Water by Complex Network Analysis}

This paper can be cited with id \texttt{rao{\_}structural{\_}2010}
\cite{rao_structural_2010}.


prior work for water features. Contrast with [2015-wetmsm].




\subsection{Simple Theory of Protein Folding Kinetics}

This paper can be cited with id \texttt{2010-pande-folding}
\cite{2010-pande-folding}.


Non-native interactions and misfolding




\subsection{Atomic-Level Characterization of the Structural Dynamics of Proteins}

This paper can be cited with id \texttt{2010-shaw-fip35-bpti}
\cite{2010-shaw-fip35-bpti}.


Simulation of fip35 ww domain: 2x 100 us. Note this was at 400K so
unfolding could be observed.

Simulation of bpti: 1ms. Note this was done with tip4p for reasons.




\subsection{Challenges in protein-folding simulations}

This paper can be cited with id \texttt{2010-schulten-challenges}
\cite{2010-schulten-challenges}.


Cited by [2014-msm-perspective] as highlighting analysis as a problem.




\subsection{Everything you wanted to know about Markov State Models but were afraid to ask}

This paper can be cited with id \texttt{2010-everything-msm-afraid-ask}
\cite{2010-everything-msm-afraid-ask}.


Review of MSMs intended for "non-experts". Obviously a little dated by now.


\subsection{Structural mechanism of C-type inactivation in K+ channels}

This paper can be cited with id \texttt{cuello2010structural}
\cite{cuello2010structural}.


\subsection{Domain Reorientation and Rotation of an Intracellular Assembly Regulate Conduction in Kir Potassium Channels}

This paper can be cited with id \texttt{clarke2010domain}
\cite{clarke2010domain}.


\subsection{Molecular Background of Leak K+ Currents: Two-Pore Domain Potassium Channels}

This paper can be cited with id \texttt{2010-k2p-review}
\cite{2010-k2p-review}.


Nice review of K2P two-pore potassium channels. They talk about the wide
variety of regulatory stimuli




\subsection{High-Throughput All-Atom Molecular Dynamics Simulations Using Distributed Computing}

This paper can be cited with id \texttt{2010-gpugrid}
\cite{2010-gpugrid}.


GPUGRID intro paper. Cite this alongside FAH. They (probably) did GPU
distributed computing before FAH.


\subsection{Principles of conduction and hydrophobic gating in K+ channels}

This paper can be cited with id \texttt{jensen2010principles}
\cite{jensen2010principles}.


\subsection{Current Status of the AMOEBA Polarizable Force Field}

This paper can be cited with id \texttt{2010-amoeba}
\cite{2010-amoeba}.


\subsection{Implementation of the CHARMM Force Field in GROMACS: Analysis of Protein Stability Effects from Correction Maps, Virtual Interaction Sites, and Water Models}

This paper can be cited with id \texttt{2010-charmm27}
\cite{2010-charmm27}.


\subsection{Web-scale k-means clustering}

This paper can be cited with id \texttt{2010-minibatch-kmeans}
\cite{2010-minibatch-kmeans}.


Clustering algorithm from sklearn admired for its speed.


\subsection{On the Approximation Quality of Markov State Models}

This paper can be cited with id \texttt{2010-msm-error}
\cite{2010-msm-error}.


\subsection{'Plenty of room' revisited}

This paper can be cited with id \texttt{2009-plenty-of-room-focus}
\cite{2009-plenty-of-room-focus}.


Editorial about [1960-plenty-of-room-at-the-bottom].




\subsection{Constructing the equilibrium ensemble of folding pathways from short off-equilibrium simulations}

This paper can be cited with id \texttt{2009-noe-equilibrium-from-short}
\cite{2009-noe-equilibrium-from-short}.


\subsection{Using generalized ensemble simulations and Markov state models to identify conformational states}

This paper can be cited with id \texttt{2009-msmbuilder1}
\cite{2009-msmbuilder1}.


This introduced the first release of MSMBuilder. You probably shouldn't
cite this unless you have a good reason to.




\subsection{Reactive flux and folding pathways in network models of coarse-grained protein dynamics}

This paper can be cited with id \texttt{2009-berezhkovskii-reactive-flux}
\cite{2009-berezhkovskii-reactive-flux}.


\subsection{Accelerating molecular dynamic simulation on graphics processing units}

This paper can be cited with id \texttt{2009-friedrichs-gpu}
\cite{2009-friedrichs-gpu}.


Probably the second instance of using GPUs for molecular dynamics. This
became [OpenMM](openmm.org).


\subsection{Long-timescale molecular dynamics simulations of protein structure and function}

This paper can be cited with id \texttt{2009-md-perspective}
\cite{2009-md-perspective}.


\subsection{Two-P-Domain (K2P) Potassium Channels: Leak Conductance Regulators of Excitability}

This paper can be cited with id \texttt{goldstein2001potassium}
\cite{goldstein2001potassium}.


\subsection{Transition Path Theory for Markov Jump Processes}

This paper can be cited with id \texttt{2009-metzner-tpt}
\cite{2009-metzner-tpt}.


Transition path theory (TPT).


\subsection{Millisecond-scale molecular dynamics simulations on Anton}

This paper can be cited with id \texttt{2009-anton}
\cite{2009-anton}.


\subsection{Fast determination of the optimal rotational matrix for macromolecular superpositions}

This paper can be cited with id \texttt{2009-theobald-rmsd}
\cite{2009-theobald-rmsd}.


This one computes the optimal rotation in addition to (specifically: after)
just computing the minimal RMSD value. It uses [2005-theobald-rmsd=14] for
finding the optimal RMSD (ie leading eigenvalue of key matrix).


\subsection{Efficient nonbonded interactions for molecular dynamics on a graphics processing unit}

This paper can be cited with id \texttt{2009-eastman-gpu}
\cite{2009-eastman-gpu}.


Optimizing below-cutoff nonbonded calculations on the GPU by tricky memory
and parallelization management. This was for OpenMM. This is not PME.


\subsection{The Membrane Potential and its Representation by a Constant Electric Field in Computer Simulations}

This paper can be cited with id \texttt{2008-roux-efield}
\cite{2008-roux-efield}.


Constant electric field




\subsection{Role of Water in Mediating the Assembly of Alzheimer Amyloid-β Aβ16−22 Protofilaments}

This paper can be cited with id \texttt{krone{\_}role{\_}2008}
\cite{krone_role_2008}.


Cited by [2015-wetmsm] where solvent is important for aggregation




\subsection{Anton, a special-purpose machine for molecular dynamics simulation}

This paper can be cited with id \texttt{2008-anton}
\cite{2008-anton}.


The seminal Anton paper. Cite this when talking about single, long
trajectories or special-purpose hardware.


\subsection{The Protein Folding Problem}

This paper can be cited with id \texttt{2008-protein-folding-problem}
\cite{2008-protein-folding-problem}.


\subsection{General purpose molecular dynamics simulations fully implemented on graphics processing units}

This paper can be cited with id \texttt{2008-anderson-gpu}
\cite{2008-anderson-gpu}.


They claim to be the first GPU accelerated MD engine too! Probably led to
HOOMD, although they don't call it that in the paper.


\subsection{Insights from the energetics of water binding at the domain-ligand interface of the Src SH2 domain}

This paper can be cited with id \texttt{fabritiis{\_}insights{\_}2008}
\cite{fabritiis_insights_2008}.


Cited by [2015-wetmsm] where solvent is treated by grid of voxels.




\subsection{Limits on Variations in Protein Backbone Dynamics from Precise Measurements of Scalar Couplings}

This paper can be cited with id \texttt{2007-scalar-coupling}
\cite{2007-scalar-coupling}.


\subsection{Motifs for molecular recognition exploiting hydrophobic enclosure in protein–ligand binding}

This paper can be cited with id \texttt{young{\_}motifs{\_}2007}
\cite{young_motifs_2007}.


Cited by [2015-wetmsm] where water is important for protein-ligand binding




\subsection{IPython: A System for Interactive Scientific Computing}

This paper can be cited with id \texttt{2007-ipython}
\cite{2007-ipython}.


\subsection{Accelerating molecular modeling applications with graphics processors}

This paper can be cited with id \texttt{2007-stone-gpu}
\cite{2007-stone-gpu}.


(Probably) the first GPU accelerated MD paper. This is for NAMD.


\subsection{Scalable Algorithms for Molecular Dynamics Simulations on Commodity Clusters}

This paper can be cited with id \texttt{2006-bowers-cluster}
\cite{2006-bowers-cluster}.


\subsection{What Is the Relation Between Slow Feature Analysis and Independent Component Analysis?}

This paper can be cited with id \texttt{doi:10.1162/neco.2006.18.10.2495}
\cite{doi:10.1162/neco.2006.18.10.2495}.


\subsection{Effects of Solvent on the Structure of the Alzheimer Amyloid-β(25–35) Peptide}

This paper can be cited with id \texttt{wei{\_}effects{\_}2006}
\cite{wei_effects_2006}.


Cited by [2015-wetmsm] where solvent is important for aggregation




\subsection{Diffusion maps, spectral clustering and reaction coordinates of dynamical systems}

This paper can be cited with id \texttt{2006-nadler-diffusion-maps}
\cite{2006-nadler-diffusion-maps}.


\subsection{Transition Networks for the Comprehensive Characterization of Complex Conformational Change in Proteins}

This paper can be cited with id \texttt{2006-noe-conf-change}
\cite{2006-noe-conf-change}.


\subsection{Nanotube Confinement Denatures Protein Helices}

This paper can be cited with id \texttt{sorin{\_}nanotube{\_}2006}
\cite{sorin_nanotube_2006}.


Cited by [2015-wetmsm] where water is important for protein stability




\subsection{Using massively parallel simulation and Markovian models to study protein folding: Examining the dynamics of the villin headpiece}

This paper can be cited with id \texttt{jayachandran{\_}using{\_}2006}
\cite{jayachandran_using_2006}.


Cited by [2015-wetmsm] as successful MSM application for folding




\subsection{Meshless Methods in Conformational Dynamics}

This paper can be cited with id \texttt{2006-meshless-msm-thesis}
\cite{2006-meshless-msm-thesis}.


Mainly concerned with lumping (PCCA) and setting up an iterative sampling
scheme, released as ZIBgridfree.

Partition of unity using Shepard's method [1968-shepard-method=117].
Definition 4.8 says these need to be positive (greater than zero) which
rules out traditional MSMs. Why?

Highlights importants of softness parameter of the shape function, which
they call alpha. They say Shepard's method with gaussian RBFs can be seen
as a generalized Voronoi Tessellation.




\subsection{A general purpose model for the condensed phases of water: TIP4P/2005}

This paper can be cited with id \texttt{2005-tip4p}
\cite{2005-tip4p}.


\subsection{Rapid calculation of RMSDs using a quaternion-based characteristic polynomial}

This paper can be cited with id \texttt{2005-theobald-rmsd}
\cite{2005-theobald-rmsd}.


Instead of doing matrix diagonalization or inversion, use netwon-raphson
root-finding on a characteristic polynomial.

Mainly builds off of [1987-horn-rmsd=5].




\subsection{Structural mechanism of the recovery stroke in the Myosin molecular motor}

This paper can be cited with id \texttt{2005-myosin-motor}
\cite{2005-myosin-motor}.


\subsection{Robust Perron cluster analysis in conformation dynamics}

This paper can be cited with id \texttt{2005-pcca}
\cite{2005-pcca}.


PCCA group states based on an MSM transition matrix. Specifically, it uses
the eigenspectrum to do the lumping. Cite this in the methods section of
your paper if you use PCCA or PCCA+.


\subsection{Scalable molecular dynamics with NAMD}

This paper can be cited with id \texttt{2005-namd}
\cite{2005-namd}.


\subsection{Hydrophobic Collapse in Multidomain Protein Folding}

This paper can be cited with id \texttt{zhou{\_}hydrophobic{\_}2004}
\cite{zhou_hydrophobic_2004}.


Cited by [2015-wetmsm] because model system for hydrophobic collapse




\subsection{Extending the treatment of backbone energetics in protein force fields: Limitations of gas-phase quantum mechanics in reproducing protein conformational distributions in molecular dynamics simulations}

This paper can be cited with id \texttt{2004-charmm27}
\cite{2004-charmm27}.


\subsection{Describing Protein Folding Kinetics by Molecular Dynamics Simulations. 1. Theory†}

This paper can be cited with id \texttt{2004-swope-msm}
\cite{2004-swope-msm}.


The first MSM paper. Gets pretty much everything right. Except they're
convinced that you need to do state exploration via NVT or NPT and then
calculate transitions by launching bespoke NVE simulations. Obviously, we
just run big NPT runs and use that for both state space exploration and
counting transitions.




\subsection{Calculating potentials of mean force from steered molecular dynamics simulations}

This paper can be cited with id \texttt{2004-pmf}
\cite{2004-pmf}.


\subsection{Using quaternions to calculate RMSD}

This paper can be cited with id \texttt{2004-dill-rmsd}
\cite{2004-dill-rmsd}.


Similar to [2005-theobald-rmsd], builds off of [1987-horn-rmsd=5]. Proves
identity with normal 3x3 methods.

Derives the derivative of RMSD wrt coordinates, although "it is well known"




\subsection{Using path sampling to build better Markovian state models: Predicting the folding rate and mechanism of a tryptophan zipper beta hairpin}

This paper can be cited with id \texttt{2004-nina-msm}
\cite{2004-nina-msm}.


\subsection{Development and testing of a general amber force field}

This paper can be cited with id \texttt{2004-gaff}
\cite{2004-gaff}.


\subsection{Modeling induced polarization with classical Drude oscillators: Theory and molecular dynamics simulation algorithm}

This paper can be cited with id \texttt{2003-drude-particles}
\cite{2003-drude-particles}.


\subsection{Topological quantum memory}

This paper can be cited with id \texttt{2002-surface-code}
\cite{2002-surface-code}.


Called the seminal work in surface code error correction by
[2017-fault-tolerant-computation], this long article seems to evaluate the
details of the surface code which were introduced in
[1997-kitaev-error-correction=4] and [1997-anyons=5].




\subsection{Hydrophobicity: Two faces of water}

This paper can be cited with id \texttt{chandler{\_}hydrophobicity{\_}2002}
\cite{chandler_hydrophobicity_2002}.


Cited by [2015-wetmsm] where water is important for hydrophobic collapse




\subsection{Using the Nyström Method to Speed Up Kernel Machines}

This paper can be cited with id \texttt{2001-nystroem}
\cite{2001-nystroem}.


\subsection{Transfer Operator Approach to Conformational Dynamics in Biomolecular Systems}

This paper can be cited with id \texttt{2001-schutte-variational}
\cite{2001-schutte-variational}.


Full treatment of transfer operator / propagator and build an MSM for a
small RNA chain.




\subsection{COMPUTING: Screen Savers of the World Unite!}

This paper can be cited with id \texttt{2000-fah}
\cite{2000-fah}.


The seminal Folding at Home paper. Cite this whenever you talk about
distributed computing or Folding at Home.

SETI@Home and distributed.net came before this.


\subsection{Identification of almost invariant aggregates in reversible nearly uncoupled Markov chains}

This paper can be cited with id \texttt{2000-pcca}
\cite{2000-pcca}.


\subsection{Human TREK2, a 2P Domain Mechano-sensitive K+Channel with Multiple Regulations by Polyunsaturated Fatty Acids, Lysophospholipids, and Gs, Gi, and GqProtein-coupled Receptors}

This paper can be cited with id \texttt{lesage2000human}
\cite{lesage2000human}.


\subsection{Molecular Dynamics of the KcsA K+ Channel in a Bilayer Membrane}

This paper can be cited with id \texttt{2000-roux-kcsa-md}
\cite{2000-roux-kcsa-md}.


They run 4ns of MD on KcsA potassium channel.




\subsection{A Direct Approach to Conformational Dynamics Based on Hybrid Monte Carlo}

This paper can be cited with id \texttt{1999-schutte-msm}
\cite{1999-schutte-msm}.


Maybe the first time conformations were discretized and a Markov operator
was made.




\subsection{Nonlinear Component Analysis as a Kernel Eigenvalue Problem}

This paper can be cited with id \texttt{1998-scholkopf-kernel-pca}
\cite{1998-scholkopf-kernel-pca}.


\subsection{Fault-tolerant quantum computation by anyons}

This paper can be cited with id \texttt{1997-anyons}
\cite{1997-anyons}.


\subsection{Stabilizer Codes and Quantum Error Correction}

This paper can be cited with id \texttt{1997-gottesman-thesis}
\cite{1997-gottesman-thesis}.


\subsection{Quantum Error Correction with Imperfect Gates}

This paper can be cited with id \texttt{1997-kitaev-error-correction}
\cite{1997-kitaev-error-correction}.


\subsection{VMD: Visual molecular dynamics}

This paper can be cited with id \texttt{1996-vmd}
\cite{1996-vmd}.


The only game in town for making movies.


\subsection{Three-dimensional Structures of Free Form and Two Substrate Complexes of an Extradiol Ring-cleavage Type Dioxygenase, the BphC Enzyme fromPseudomonassp. Strain KKS102}

This paper can be cited with id \texttt{1996-bphc-structure}
\cite{1996-bphc-structure}.


\subsection{Knowledge-based protein secondary structure assignment}

This paper can be cited with id \texttt{1995-stride}
\cite{1995-stride}.


VMD wants you to cite this for secondary structure prediction


\subsection{Cα-based torsion angles: A simple tool to analyze protein conformational changes}

This paper can be cited with id \texttt{1995-alpha-carbon}
\cite{1995-alpha-carbon}.


Alpha carbon featurization


\subsection{Fast Parallel Algorithms for Short-Range Molecular Dynamics}

This paper can be cited with id \texttt{1995-plimpton}
\cite{1995-plimpton}.


\subsection{Separation of a mixture of independent signals using time delayed correlations}

This paper can be cited with id \texttt{doi:10.1103/PhysRevLett.72.3634}
\cite{doi:10.1103/PhysRevLett.72.3634}.


\subsection{THE weighted histogram analysis method for free-energy calculations on biomolecules. I. The method}

This paper can be cited with id \texttt{1992-wham}
\cite{1992-wham}.


Wham reweighting algorithm, perhaps used after umbrella sampling.


\subsection{The energy landscapes and motions of proteins}

This paper can be cited with id \texttt{1991-complex-protein-energy-landscapes}
\cite{1991-complex-protein-energy-landscapes}.


Cited by [2011-prinz] to say that there are many metastable states and many
timescales.




\subsection{Two types of inactivation in Shaker K+ channels: Effects of alterations in the carboxy-terminal region}

This paper can be cited with id \texttt{hoshi1991two}
\cite{hoshi1991two}.


\subsection{On the orthogonal transformation used for structural comparisons}

This paper can be cited with id \texttt{1989-kearsley-rmsd}
\cite{1989-kearsley-rmsd}.


[2005-theobald-rmsd] cites for quaternion RMSD


\subsection{Modern Quantum Chemistry: Introduction to Advanced Electronic Structure Theory}

This paper can be cited with id \texttt{1989-szabo-ostlund-qm}
\cite{1989-szabo-ostlund-qm}.


Cited by [2013-noe-variational] for Rayleigh variational method.




\subsection{Computer Simulation of Liquids}

This paper can be cited with id \texttt{1989-computer-simulation-of-liquids}
\cite{1989-computer-simulation-of-liquids}.


\subsection{A note on the rotational superposition problem}

This paper can be cited with id \texttt{1988-diamond-rmsd}
\cite{1988-diamond-rmsd}.


[2005-theobald-rmsd] cites for quaternion RMSD




\subsection{An efficient newton-like method for molecular mechanics energy minimization of large molecules}

This paper can be cited with id \texttt{1987-minimization}
\cite{1987-minimization}.


\subsection{Closed-form solution of absolute orientation using unit quaternions}

This paper can be cited with id \texttt{1987-horn-rmsd}
\cite{1987-horn-rmsd}.


[2005-theobald-rmsd] cites for quaternion RMSD




\subsection{Measurements of Macroscopic Quantum Tunneling out of the Zero-Voltage State of a Current-Biased Josephson Junction}

This paper can be cited with id \texttt{1985-macroscopic-quantum-tunneling}
\cite{1985-macroscopic-quantum-tunneling}.


\subsection{Hydrogen bonding in globular proteins}

This paper can be cited with id \texttt{1984-baker-hubbard}
\cite{1984-baker-hubbard}.


Hydrogen bond determination




\subsection{CHARMM: A program for macromolecular energy, minimization, and dynamics calculations}

This paper can be cited with id \texttt{1983-charmm}
\cite{1983-charmm}.


\subsection{A solution for the best rotation to relate two sets of vectors}

This paper can be cited with id \texttt{1978-kabsch-rmsd}
\cite{1978-kabsch-rmsd}.


[2005-theobald-rmsd] says this suffers from rotoinversions because it uses
3x3 rotations instead of 4x4 quaternions.




\subsection{Environment and exposure to solvent of protein atoms. Lysozyme and insulin}

This paper can be cited with id \texttt{1973-shrake-rupley}
\cite{1973-shrake-rupley}.


Solvation




\subsection{ff14SB: improving the accuracy of protein side chain and backbone parameters from ff99SB}

This paper can be cited with id \texttt{maier2015ff14sb}
\cite{maier2015ff14sb}.


\subsection{Understanding Protein Dynamics with L1-Regularized Reversible Hidden Markov Models}

This paper can be cited with id \texttt{2014-mcgibbon-hmm}
\cite{2014-mcgibbon-hmm}.


Use hidden markov models instead of discrete state MSMs.




\subsection{UCSF Chimera-a visualization system for exploratory research and analysis}

This paper can be cited with id \texttt{pettersen2004ucsf}
\cite{pettersen2004ucsf}.


\subsection{Stochastic Processes in Physics and Chemistry}

This paper can be cited with id \texttt{2987-van-kampen-book}
\cite{2987-van-kampen-book}.


\subsection{Steered molecular dynamics}

This paper can be cited with id \texttt{izrailev1999steered}
\cite{izrailev1999steered}.


[2015-wetmsm] cited this for methods: steered MD




\subsection{Protein kinases: evolution of dynamic regulatory proteins}

This paper can be cited with id \texttt{taylor2011protein}
\cite{taylor2011protein}.


Cited in [2015-wetmsm] intro.




\subsection{Particle mesh Ewald: An Nlog (N) method for Ewald sums in large systems}

This paper can be cited with id \texttt{darden1993particle}
\cite{darden1993particle}.


[2015-wetmsm] cited this for methods: pme


\subsection{Levinthal's paradox}

This paper can be cited with id \texttt{1992-levinthal-paradox}
\cite{1992-levinthal-paradox}.


Conformational space is huge, but proteins can fold very fast.


\subsection{Large conformational changes in proteins: signaling and other functions}

This paper can be cited with id \texttt{grant2010large}
\cite{grant2010large}.


Cited in [2015-wetmsm] intro.




\subsection{Landmark Kernel tICA For Conformational Dynamics}

This paper can be cited with id \texttt{2017-lktica}
\cite{2017-lktica}.


\subsection{LINCS: a linear constraint solver for molecular simulations}

This paper can be cited with id \texttt{hess1997lincs}
\cite{hess1997lincs}.


[2015-wetmsm] cited this for methods: lincs




\subsection{Improved side-chain torsion potentials for the Amber ff99SB protein force field}

This paper can be cited with id \texttt{lindorff2010improved}
\cite{lindorff2010improved}.


[2015-wetmsm] cited this for methods: amber99sb-ildn




\subsection{GROMACS 4: Algorithms for highly efficient, load-balanced, and scalable molecular simulation}

This paper can be cited with id \texttt{hess2008gromacs}
\cite{hess2008gromacs}.


[2015-wetmsm] cited this for methods: gromacs




\subsection{Dynamic personalities of proteins}

This paper can be cited with id \texttt{henzler2007dynamic}
\cite{henzler2007dynamic}.


Cited in [2015-wetmsm] intro.




\subsection{Complex pathways in folding of protein G explored by simulation and experiment}

This paper can be cited with id \texttt{lapidus2014complex}
\cite{lapidus2014complex}.


Cited by [2015-wetmsm] where they discard water.




\subsection{Comparison of simple potential functions for simulating liquid water}

This paper can be cited with id \texttt{jorgensen1983comparison}
\cite{jorgensen1983comparison}.


[2015-wetmsm] cited this for methods: tip3p.




\subsection{An Introduction to Markov State Models and Their Application to Long Timescale Molecular Simulation}

This paper can be cited with id \texttt{2014-msm-book}
\cite{2014-msm-book}.


\subsection{Amber 14}

This paper can be cited with id \texttt{case2014amber}
\cite{case2014amber}.


\subsection{A Fast 3 x N Matrix Multiply Routine for Calculation of Protein RMSD}

This paper can be cited with id \texttt{2014-haque-fast-rmsd}
\cite{2014-haque-fast-rmsd}.


\subsection{}

This paper can be cited with id \texttt{1968-levinthal-paradox}
\cite{1968-levinthal-paradox}.


Taken from [Nature's protein folding
focus](http://www.nature.com/nsmb/focus/proteinfolding/classics/early.html)

Among the most widely cited-yet least read-papers in the field, partly
owing to the difficulties in getting hold of them, Cyrus Levinthal used a
simple model to show that a typical polypeptide chain cannot fold through
an unbiased search of all conformational space on a reasonable timescale.
This is commonly referred to as the "Levinthal's paradox", and led to the
concept that proteins fold along discrete pathways. The first paper
presents this idea and is usually cited, but the model is actually
presented in the second one. Although the model was later shown to be
overly simplistic, the work had a crucial role in directing the search and
characterization of intermediate states.




\subsection{A two-dimensional interpolation function for irregularly-spaced data}

This paper can be cited with id \texttt{1968-shepard-method}
\cite{1968-shepard-method}.


Method for interpolation confusingly cited by [2006-meshless-msm-thesis]. I
guess he introduces weightedsum(inverse distances) / sum(inverse
distances). And instead of inverse distances, you can choose whatever
function you want.




\subsection{There's Plenty of Room at the Bottom}

This paper can be cited with id \texttt{1960-plenty-of-room-at-the-bottom}
\cite{1960-plenty-of-room-at-the-bottom}.


\subsection{The potassium permeability of a giant nerve fibre}

This paper can be cited with id \texttt{hodgkin1955potassium}
\cite{hodgkin1955potassium}.





\bibliography{all.bib}

\end{document}