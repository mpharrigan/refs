

% This file is autogenerated. Don't edit it
\documentclass{article}

\usepackage{amsmath}              % need for subequations
\usepackage{amssymb}              % for symbols
\usepackage{hyperref}             % put links on stuff
\usepackage{cleveref}             % use for referencing figures/equations
                                  % hyperref must come before cleverref
\usepackage{natbib}               % biblio
\usepackage{color}                % use if color is used in text
\usepackage{soul}                 % hyphenation, strikethrough
\usepackage{graphicx}
\usepackage{parskip}

\bibliographystyle{unsrt}

\begin{document}

\title{A collection of papers}
\date{\today}

\author{gitbib}

\maketitle

\section{Section title tbd}
\subsection{tICA-Metadynamics: Accelerating Metadynamics by Using Kinetically Selected Collective Variables}

This paper can be cited with id \texttt{2017-tica-metadynamics}
\cite{2017-tica-metadynamics}.


\subsection{Variational Koopman models: Slow collective variables and molecular kinetics from short off-equilibrium simulations}

This paper can be cited with id \texttt{2016-noe-reversible-tica}
\cite{2016-noe-reversible-tica}.


Provides a better way of "symetrizing" tICA correlation matrix. In tICA,
you assume that the dynamics are reversible. When we're learning from
finite data, this reversibility isn't respected. Historically, you take
your correlation matrix, add its transpose, and divide by two. This is an
especially poor approximation if you have many short trajectories. This
paper is analogous to the MLE method for symetrizing MSM counts matrices.




\subsection{Set-free Markov state model building}

This paper can be cited with id \texttt{2017-set-free-msm}
\cite{2017-set-free-msm}.


Collection of m "base points" they make a gaussian RBF of distances to base
points. They normalize it to unity. This is the softmax function, but they
don't call it that.

They add base points adaptively and use PCCA+ to lump.

Note that they have stopped calling this "meshless" or "mesh-free",
probably because the regular MSM is also meshless. Now the abstract says
"This kind of meshless discretization..."




\subsection{Identification of simple reaction coordinates from complex dynamics}

This paper can be cited with id \texttt{2016-sparsetica}
\cite{2016-sparsetica}.


The authors argue for a definition of the reaction coordinate as the
projection on the dominant eigenfunciton of the propogator. Notably, they
say that path-based coordinates are no good, because progress is only
defined along the path. They argue that the coordinate shouldn't depend on
start and end points. They say the projection should be maximally
predictive. This means finding the slowest modes. They note
[2006-nadler-diffusion-maps=61] and [2011-rohrdanz-diffusion-maps=62] have
used this definition.

They go on to show tICA finds this reaction coordinate. To make tICA more
interpretable, they develop an algorithm for introducing a sparsity
pattern. It's a pseudo-l0 regularization (made smooth so the optimization
works).

They also use a unique dihedral featurization: instead of taking the sine
and cosine to get around periodicity concerns; they project the values on a
bunch of evenly spaced von-mises (periodic gaussians) distributions around
the unit circle. Each dihedral is expanded into several numbers. It's like
a smooth histogramming. This probably won't work as the number of dihedrals
gets large (too many features).




\subsection{Optimized parameter selection reveals trends in Markov state models for protein folding}

This paper can be cited with id \texttt{2016-husic-msms}
\cite{2016-husic-msms}.


The authors perform GMRQ cross validation on the twelve
[2011-larsen-folding] folding trajectories to give guidelines for MSM
construction.

They present a flowchart for MSM construction that shows the three paths
towards clustering: from an rmsd distance metric, from features, or from
tICA learned on features.

They introduce GMRQ cross validation in the tradition of
[2015-mcgibbon-gmrq=44].

They present results but stress that you have to do your own cross
validataion to be sure. Some conclusions include: 1. tICA and PCA are
better than direct clustering of features 2. when using tica, you can use
kcenters, kmeans, or minibatch kmeans to the same effect

On one protein (2p6j) they look at all different features and show that
they vary a lot. It's unfortunate that this was only done on one protein.




\subsection{Commute Maps: Separating Slowly Mixing Molecular Configurations for Kinetic Modeling}

This paper can be cited with id \texttt{2016-commute-maps}
\cite{2016-commute-maps}.


Scale tIC coordinates by a function of the timescale. See also
[2015-kinetic-mapping].




\subsection{Notes on the Theory of Markov Chains in a Continuous State Space}

This paper can be cited with id \texttt{2016-mcgibbon-notes}
\cite{2016-mcgibbon-notes}.


\subsection{Estimation and uncertainty of reversible Markov models}

This paper can be cited with id \texttt{2015-uncertainty-estimation}
\cite{2015-uncertainty-estimation}.


Reversible estimates for MSMs


\subsection{A Basis Set for Peptides for the Variational Approach to Conformational Kinetics}

This paper can be cited with id \texttt{2015-amino-acid-basis}
\cite{2015-amino-acid-basis}.


Authors simulate individual (capped) amino acids for 1us / each and
construct (mini-)MSMs on each one. They use the outerproduct of these
mini-MSMs to serve as a basis set for peptides. MiniMSMs are on a grid in
phi-psi angles. Since each miniMSM has approx 3 modes, the full basis would
be 3^(N), which is way too big! They call the second and third modes
"excited states" and use a basis set that contains a singly-exited residue.
E.g. 11111 + [ [21111, 121111, 112111, 111211, ...] ].

Alanine preceded by a proline is taken as a special case.




\subsection{Efficient maximum likelihood parameterization of continuous-time Markov processes}

This paper can be cited with id \texttt{2015-ratematrix}
\cite{2015-ratematrix}.


\subsection{A critical appraisal of Markov state models}

This paper can be cited with id \texttt{2015-schutte-msm}
\cite{2015-schutte-msm}.


Transfer operator [1999-schutte-msm=1].

Coarse grain MSM states [2000-pcca=2] [2005-pcca=3].

Meshless MSMs [2006-meshless-msm-thesis=24] [2011-meshless-msm=32]
[2011-meshless-msm=33]. Wikipedia says these are also called "meshfree"
methods.




\subsection{Kinetic distance and kinetic maps from molecular dynamics simulation}

This paper can be cited with id \texttt{2015-kinetic-mapping}
\cite{2015-kinetic-mapping}.


Scale tIC coorindates by the eigenvalue. See also [2016-commute-maps].




\subsection{Variational cross-validation of slow dynamical modes in molecular kinetics}

This paper can be cited with id \texttt{2015-mcgibbon-gmrq}
\cite{2015-mcgibbon-gmrq}.


\subsection{Conserve Water: A Method for the Analysis of Solvent in Molecular Dynamics}

This paper can be cited with id \texttt{2015-wetmsm}
\cite{2015-wetmsm}.


Solvent-shells featurization for including solvent in MSM construction.




\subsection{Gaussian Markov transition models of molecular kinetics}

This paper can be cited with id \texttt{2015-gaussian-msms}
\cite{2015-gaussian-msms}.


Variational method [2013-noe-variational=26] [2014-nuske-variational=27].

MSM is variational with step functions [2013-noe-variational=26].

"Markov transition models (MTMs)", specifically Gaussian mixtures (GMTM).




\subsection{Markov State Models Provide Insights into Dynamic Modulation of Protein Function}

This paper can be cited with id \texttt{2015-shukla-msm-review}
\cite{2015-shukla-msm-review}.


\subsection{Modeling Molecular Kinetics with tICA and the Kernel Trick}

This paper can be cited with id \texttt{2015-schwantes-ktica}
\cite{2015-schwantes-ktica}.


They introduce kernel tICA as an extension to tICA. This is useful to get
non-linear solutions to the tICA equation. They claim you can estimate
eigenprocesses without building an MSM.

They briefly introduce the transfer operator. They introduce the
variational principle of conformation dynamics per [2011-prinz=25]. They
introduce tICA as maximizing the autocorrelation. They say that solutions
to tICA are the same as solutions to the variational problem per
[2013-noe-tica=28]. Linearity makes them crude solutions.

They explain that a natural approach to introduce non-linearity is to
expand the original representation into a higher dimensional space and do
tICA there. They say this is impractical. The expanded space probably has
to be huge. You can perform analysis in the big representation without
explicitly representing it by using the "kernel trick". They reproduce an
example of the kernel trick from [1998-scholkopf-kernel-pca=39].

They re-write the tICA problem only in terms of inner products so you can
apply the kernel trick. They introduce normalization. They choose a
gaussian kernel. They simulate a four-well potential, muller potential,
alanine dipeptide, and fip35ww. They need to do MLE cross validation over
parameters (kernel width and regularization strength).

This uses so much RAM! Huge matrices to solve (that scale with the amount
of data!!)




\subsection{Perspective: Markov models for long-timescale biomolecular dynamics}

This paper can be cited with id \texttt{2014-msm-perspective}
\cite{2014-msm-perspective}.


Very good perspective on the importance of analysis (particularly MSM
analysis) for understanding large, modern MD datasets. Money quote: "we
believe that quantitative analysis has increasingly become a limiting
factor in the application of MD"


\subsection{Statistical Model Selection for Markov Models of Biomolecular Dynamics}

This paper can be cited with id \texttt{2014-mcgibbon-bic}
\cite{2014-mcgibbon-bic}.


This is before [2015-mcgibbon-gmrq] GRMQ cross-validation. They explicitly
find the volume of voronoi cells (in low number of tIC space) to find a
likelihood. They use AIC/BIC to find the number of states to use. Finding
volumes is tough and you still can't compare across protocols (so you can
basically only scan number of states or clustering method), but! this was
the first paper to seriously suggest using a smaller number of states to
avoid overfitting.




\subsection{Variational Approach to Molecular Kinetics}

This paper can be cited with id \texttt{2014-nuske-variational}
\cite{2014-nuske-variational}.


This paper is largely redundant with [2013-noe-variational=65]. They cite
it as such: "Following the recently introduced variational principle for
metastable stochastic processes,(65) we propose a variational approach to
molecular kinetics."

They perform their variational approach on 2- and 10-alanine in addition to
1D potentials.

This comes after tICA and cites [2013-schwantes-tica=57] and
[2013-noe-tica=58] in the intro, but does nothing further with it. In
particular, they don't note that tICA is just another choice of basis set.

They cite their error paper [2010-msm-error=55].




\subsection{Markov state models of biomolecular conformational dynamics}

This paper can be cited with id \texttt{2014-chodera-msm}
\cite{2014-chodera-msm}.


Overview of MSMs, stressing eigensystem and variational approach. Includes
further reading suggestions.


\subsection{Spectral Rate Theory for Two-State Kinetics}

This paper can be cited with id \texttt{2014-prinz-rate}
\cite{2014-prinz-rate}.


\subsection{Cloud-based simulations on Google Exacycle reveal ligand modulation of GPCR activation pathways}

This paper can be cited with id \texttt{2014-kohlhoff-exacycle}
\cite{2014-kohlhoff-exacycle}.


They used Google's Exacycle to do these simulations. You can cite this for
more examples of distributed computing. It's ostensibly about GPCRs.


\subsection{Projected and hidden Markov models for calculating kinetics and metastable states of complex molecules}

This paper can be cited with id \texttt{2013-noe-hmm}
\cite{2013-noe-hmm}.


\subsection{Rapid Exploration of Configuration Space with Diffusion-Map-Directed Molecular Dynamics}

This paper can be cited with id \texttt{2013-diffusion-map-sampling}
\cite{2013-diffusion-map-sampling}.


Use diffusion maps to run umberlla sampling




\subsection{Learning Kinetic Distance Metrics for Markov State Models of Protein Conformational Dynamics}

This paper can be cited with id \texttt{2013-mcgibbon-kdml}
\cite{2013-mcgibbon-kdml}.


Learn scaling of coordinates to better approximate kinetics? Redundant with
tICA.




\subsection{Identification of slow molecular order parameters for Markov model construction}

This paper can be cited with id \texttt{2013-noe-tica}
\cite{2013-noe-tica}.


The Noe group introduces tica concomitantly with [2013-schwantes-tica].
They use the variational approach from [2013-noe-variational] to derive the
tICA equation. They cite a 2001 book about independent component analysis.


\subsection{Improvements in Markov State Model Construction Reveal Many Non-Native Interactions in the Folding of NTL9}

This paper can be cited with id \texttt{2013-schwantes-tica}
\cite{2013-schwantes-tica}.


The Pande group introduces tica concomitantly with [2013-noe-tica]. This
paper uses PCA as inspiration and cites signal processing literature.


\subsection{To milliseconds and beyond: challenges in the simulation of protein folding}

This paper can be cited with id \texttt{2013-milliseconds-folding}
\cite{2013-milliseconds-folding}.


The state of folding simulations as it was in 2013. Has a nice plot of
folding time by year by lab. Discusses the state of MSMs for analysis.
Maybe cite this if you're doing folding or want to talk about how
timescales are getting longer (and analysis is getting harder). The
references include "recommended readings", which is nice.


\subsection{A Variational Approach to Modeling Slow Processes in Stochastic Dynamical Systems}

This paper can be cited with id \texttt{2013-noe-variational}
\cite{2013-noe-variational}.


I think the point of this versus [2014-nuske-variational] is to be "protein
agnostic". They allude to proteins, but say this is more general. Their
example is a double-well potential.

They introduce the propogator formalism and stipulate that dynamics can be
seperated into "fast" and "slow" components. In contrast to a quantum
mechanics Hamiltonian, we don't know the propogator here. You have to infer
it from data.

They claim the error bound derived in [2010-msm-error=34] is not
constructive, whereas this method *is* constructive.

Math section heavily cites [2010-msm-error=34].

They adapt the Rayleigh variational principle from quantum mechanics, and
cite [1989-szabo-ostlund-qm=43]. They show that the autocorrelation of the
true first dynamical eigenfunction is its eigenvalue, and an estimate of
the first dynamical eigenfunction necessarily has a smaller eigenvalue.
This sets the variational bound. In terms of names that don't seem to be
used now that we're in the future: the Ritz method is for when you have no
overlap integrals (e.g. MSMs) and the Roothan-Hall method is for when you
do (tICA).

They put it to the test on a double well potential. They use indicator
basis functions to make an MSM; hermite basis functions so they still have
no overlap integrals, but smooth functions; and gaussian basis functions
(with overlap integrals). This must have come before tICA because there is
no mention made of it, even though it would fit in nicely.




\subsection{A Meshless Discretization Method for Markov State Models Applied to Explicit Water Peptide Folding Simulations}

This paper can be cited with id \texttt{2013-meshless-msm}
\cite{2013-meshless-msm}.


Soften the hard clustering [2006-meshless-msm-thesis=37].

Cite Shepard's approach [1968-shepard-method=30] like
[2006-meshless-msm-thesis] does to introduce the softmax function as basis
functions with softness parameter alpha. Note that this is not Shepard's
method.

They frame everything in the context of lumping and PCCA+ and use
ZIBgridfree to simulate trialanine faster than unbiased (100ns vs 10ns).




\subsection{Distribution of Reciprocal of Interatomic Distances: A Fast Structural Metric}

This paper can be cited with id \texttt{2012-drid}
\cite{2012-drid}.


A unique featurization that encodes each atom by the first ~3 moments of
its distribution of 1/distance to every other atom. Cite this if you use
this featurization.


\subsection{Nyström method vs random fourier features: A theoretical and empirical comparison}

This paper can be cited with id \texttt{2012-nystroem}
\cite{2012-nystroem}.


\subsection{Estimating the Eigenvalue Error of Markov State Models}

This paper can be cited with id \texttt{2012-eigenvalue-error}
\cite{2012-eigenvalue-error}.


\subsection{Soft Versus Hard Metastable Conformations in Molecular Simulations}

This paper can be cited with id \texttt{2011-meshless-msm}
\cite{2011-meshless-msm}.


They note MSM is a meshfree method with characteristic basis functions.

They define a "hard decomposition" in the obvious way. They define a "soft
decomposition" also as a partitioning of unity, but allowing overlap.

They still do PCCA+ and it's unclear what shape function they're using for
softness. As an example, they lump 504 soft states into 5 macrostates of
alanine dipeptide, sometimes spelled alanin dipeptid




\subsection{Markov models of molecular kinetics: Generation and validation}

This paper can be cited with id \texttt{2011-prinz}
\cite{2011-prinz}.


Fantastic in-depth intro to MSMs. Figure 1 in this paper is necessary for
understanding eigenvectors. This defines and relates the propogator and
transfer operator. This shows how we compute timescales from eigenvectors.
This discusess state decomposition error and shows that many states are
needed in transition regions.

quote: it is clear that a “sufficiently fine” partitioning will be able to
resolve “sufficient” detail [2010-msm-error].

Cites [2004-nina-msm] for use of the term "MSM".




\subsection{Determination of reaction coordinates via locally scaled diffusion map}

This paper can be cited with id \texttt{2011-rohrdanz-diffusion-maps}
\cite{2011-rohrdanz-diffusion-maps}.


\subsection{Slow dynamics in protein fluctuations revealed by time-structure based independent component analysis: The case of domain motions}

This paper can be cited with id \texttt{2011-japan-tica}
\cite{2011-japan-tica}.


Probably the first application of tICA to MD.


\subsection{Simple Theory of Protein Folding Kinetics}

This paper can be cited with id \texttt{2010-pande-folding}
\cite{2010-pande-folding}.


Non-native interactions and misfolding




\subsection{Challenges in protein-folding simulations}

This paper can be cited with id \texttt{2010-schulten-challenges}
\cite{2010-schulten-challenges}.


Cited by [2014-msm-perspective] as highlighting analysis as a problem.




\subsection{Everything you wanted to know about Markov State Models but were afraid to ask}

This paper can be cited with id \texttt{2010-everything-msm-afraid-ask}
\cite{2010-everything-msm-afraid-ask}.


Review of MSMs intended for "non-experts". Obviously a little dated by now.


\subsection{High-Throughput All-Atom Molecular Dynamics Simulations Using Distributed Computing}

This paper can be cited with id \texttt{2010-gpugrid}
\cite{2010-gpugrid}.


GPUGRID intro paper. Cite this alongside FAH. They (probably) did GPU
distributed computing before FAH.


\subsection{On the Approximation Quality of Markov State Models}

This paper can be cited with id \texttt{2010-msm-error}
\cite{2010-msm-error}.


\subsection{'Plenty of room' revisited}

This paper can be cited with id \texttt{2009-plenty-of-room-focus}
\cite{2009-plenty-of-room-focus}.


Editorial about [1960-plenty-of-room-at-the-bottom].




\subsection{Accelerating molecular dynamic simulation on graphics processing units}

This paper can be cited with id \texttt{2009-friedrichs-gpu}
\cite{2009-friedrichs-gpu}.


Probably the second instance of using GPUs for molecular dynamics. This
became [OpenMM](openmm.org).


\subsection{Long-timescale molecular dynamics simulations of protein structure and function}

This paper can be cited with id \texttt{2009-md-perspective}
\cite{2009-md-perspective}.


\subsection{Efficient nonbonded interactions for molecular dynamics on a graphics processing unit}

This paper can be cited with id \texttt{2009-eastman-gpu}
\cite{2009-eastman-gpu}.


Optimizing below-cutoff nonbonded calculations on the GPU by tricky memory
and parallelization management. This was for OpenMM. This is not PME.


\subsection{Anton, a special-purpose machine for molecular dynamics simulation}

This paper can be cited with id \texttt{2008-anton}
\cite{2008-anton}.


The seminal Anton paper. Cite this when talking about single, long
trajectories or special-purpose hardware.


\subsection{The Protein Folding Problem}

This paper can be cited with id \texttt{2008-protein-folding-problem}
\cite{2008-protein-folding-problem}.


\subsection{General purpose molecular dynamics simulations fully implemented on graphics processing units}

This paper can be cited with id \texttt{2008-anderson-gpu}
\cite{2008-anderson-gpu}.


They claim to be the first GPU accelerated MD engine too! Probably led to
HOOMD, although they don't call it that in the paper.


\subsection{Accelerating molecular modeling applications with graphics processors}

This paper can be cited with id \texttt{2007-stone-gpu}
\cite{2007-stone-gpu}.


(Probably) the first GPU accelerated MD paper. This is for NAMD.


\subsection{Diffusion maps, spectral clustering and reaction coordinates of dynamical systems}

This paper can be cited with id \texttt{2006-nadler-diffusion-maps}
\cite{2006-nadler-diffusion-maps}.


\subsection{Meshless Methods in Conformational Dynamics}

This paper can be cited with id \texttt{2006-meshless-msm-thesis}
\cite{2006-meshless-msm-thesis}.


Mainly concerned with lumping (PCCA) and setting up an iterative sampling
scheme, released as ZIBgridfree.

Partition of unity using Shepard's method [1968-shepard-method=117].
Definition 4.8 says these need to be positive (greater than zero) which
rules out traditional MSMs. Why?

Highlights importants of softness parameter of the shape function, which
they call alpha. They say Shepard's method with gaussian RBFs can be seen
as a generalized Voronoi Tessellation.




\subsection{Robust Perron cluster analysis in conformation dynamics}

This paper can be cited with id \texttt{2005-pcca}
\cite{2005-pcca}.


PCCA group states based on an MSM transition matrix. Specifically, it uses
the eigenspectrum to do the lumping. Cite this in the methods section of
your paper if you use PCCA or PCCA+.


\subsection{Describing Protein Folding Kinetics by Molecular Dynamics Simulations. 1. Theory†}

This paper can be cited with id \texttt{2004-swope-msm}
\cite{2004-swope-msm}.


The first MSM paper. Gets pretty much everything right. Except they're
convinced that you need to do state exploration via NVT or NPT and then
calculate transitions by launching bespoke NVE simulations. Obviously, we
just run big NPT runs and use that for both state space exploration and
counting transitions.




\subsection{Using path sampling to build better Markovian state models: Predicting the folding rate and mechanism of a tryptophan zipper beta hairpin}

This paper can be cited with id \texttt{2004-nina-msm}
\cite{2004-nina-msm}.


\subsection{Using the Nyström Method to Speed Up Kernel Machines}

This paper can be cited with id \texttt{2001-nystroem}
\cite{2001-nystroem}.


\subsection{Transfer Operator Approach to Conformational Dynamics in Biomolecular Systems}

This paper can be cited with id \texttt{2001-schutte-variational}
\cite{2001-schutte-variational}.


Full treatment of transfer operator / propagator and build an MSM for a
small RNA chain.




\subsection{COMPUTING: Screen Savers of the World Unite!}

This paper can be cited with id \texttt{2000-fah}
\cite{2000-fah}.


The seminal Folding at Home paper. Cite this whenever you talk about
distributed computing or Folding at Home.

SETI@Home and distributed.net came before this.


\subsection{Identification of almost invariant aggregates in reversible nearly uncoupled Markov chains}

This paper can be cited with id \texttt{2000-pcca}
\cite{2000-pcca}.


\subsection{A Direct Approach to Conformational Dynamics Based on Hybrid Monte Carlo}

This paper can be cited with id \texttt{1999-schutte-msm}
\cite{1999-schutte-msm}.


Maybe the first time conformations were discretized and a Markov operator
was made.




\subsection{Nonlinear Component Analysis as a Kernel Eigenvalue Problem}

This paper can be cited with id \texttt{1998-scholkopf-kernel-pca}
\cite{1998-scholkopf-kernel-pca}.


\subsection{The energy landscapes and motions of proteins}

This paper can be cited with id \texttt{1991-complex-protein-energy-landscapes}
\cite{1991-complex-protein-energy-landscapes}.


Cited by [2011-prinz] to say that there are many metastable states and many
timescales.




\subsection{Modern Quantum Chemistry: Introduction to Advanced Electronic Structure Theory}

This paper can be cited with id \texttt{1989-szabo-ostlund-qm}
\cite{1989-szabo-ostlund-qm}.


Cited by [2013-noe-variational] for Rayleigh variational method.




\subsection{Understanding Protein Dynamics with L1-Regularized Reversible Hidden Markov Models}

This paper can be cited with id \texttt{2014-mcgibbon-hmm}
\cite{2014-mcgibbon-hmm}.


Use hidden markov models instead of discrete state MSMs.




\subsection{Stochastic Processes in Physics and Chemistry}

This paper can be cited with id \texttt{2987-van-kampen-book}
\cite{2987-van-kampen-book}.


\subsection{Levinthal's paradox}

This paper can be cited with id \texttt{1992-levinthal-paradox}
\cite{1992-levinthal-paradox}.


Conformational space is huge, but proteins can fold very fast.


\subsection{Landmark Kernel tICA For Conformational Dynamics}

This paper can be cited with id \texttt{2017-lktica}
\cite{2017-lktica}.


\subsection{}

This paper can be cited with id \texttt{1968-levinthal-paradox}
\cite{1968-levinthal-paradox}.


Taken from [Nature's protein folding
focus](http://www.nature.com/nsmb/focus/proteinfolding/classics/early.html)

Among the most widely cited-yet least read-papers in the field, partly
owing to the difficulties in getting hold of them, Cyrus Levinthal used a
simple model to show that a typical polypeptide chain cannot fold through
an unbiased search of all conformational space on a reasonable timescale.
This is commonly referred to as the "Levinthal's paradox", and led to the
concept that proteins fold along discrete pathways. The first paper
presents this idea and is usually cited, but the model is actually
presented in the second one. Although the model was later shown to be
overly simplistic, the work had a crucial role in directing the search and
characterization of intermediate states.




\subsection{A two-dimensional interpolation function for irregularly-spaced data}

This paper can be cited with id \texttt{1968-shepard-method}
\cite{1968-shepard-method}.


Method for interpolation confusingly cited by [2006-meshless-msm-thesis]. I
guess he introduces weightedsum(inverse distances) / sum(inverse
distances). And instead of inverse distances, you can choose whatever
function you want.




\subsection{There's Plenty of Room at the Bottom}

This paper can be cited with id \texttt{1960-plenty-of-room-at-the-bottom}
\cite{1960-plenty-of-room-at-the-bottom}.





\bibliography{msm.bib}

\end{document}