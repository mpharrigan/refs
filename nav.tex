

% This file is autogenerated. Don't edit it
\documentclass{article}

\usepackage{amsmath}              % need for subequations
\usepackage{amssymb}              % for symbols
\usepackage{hyperref}             % put links on stuff
\usepackage{cleveref}             % use for referencing figures/equations
                                  % hyperref must come before cleverref
\usepackage{natbib}               % biblio
\usepackage{color}                % use if color is used in text
\usepackage{soul}                 % hyphenation, strikethrough
\usepackage{graphicx}
\usepackage{parskip}

\bibliographystyle{unsrt}

\begin{document}

\title{A collection of papers}
\date{\today}

\author{gitbib}

\maketitle

\section{Section title tbd}
\subsection{Cryo-EM Structure of the Open Human  Ether-à-go-go -Related K +  Channel hERG}

This paper can be cited with id \texttt{2017-herg-structure}
\cite{2017-herg-structure}.


\subsection{Structures of closed and open states of a voltage-gated sodium channel}

This paper can be cited with id \texttt{2017-navab}
\cite{2017-navab}.


Open PD/VSD (5vb8) and closed PD/VSD/CTD (5vb2)




\subsection{The complete structure of an activated open sodium channel}

This paper can be cited with id \texttt{2017-open-full-navms}
\cite{2017-open-full-navms}.


First full-length, open conformation. [2017-navab] has a full-length
closed. This is NavMs and that is NavAb.




\subsection{Structure of a eukaryotic voltage-gated sodium channel at near-atomic resolution}

This paper can be cited with id \texttt{2017-euk-navpas}
\cite{2017-euk-navpas}.


First eukaryotic structure. CryoEM of cockroach NaV. Might not be a NaV. No
elecrophysiology can be performed. Was originally called PaFPC1, they
renamed it NaVPaS




\subsection{Simulating the Activation of Voltage Sensing Domain for a Voltage-Gated Sodium Channel Using Polarizable Force Field}

This paper can be cited with id \texttt{2017-vsd-only-pmf}
\cite{2017-vsd-only-pmf}.


They simulated only one domain. They took the NavAb vsd and connected it
via molecular dynamics to a homology model to a [sea
squirt](https://en.wikipedia.org/wiki/Ciona_intestinalis) vsd. They used a
polarizable force-field, which is more expensive than normal (fixed charge)
forcefields but that might be important for something so intertwined with
moving electrical charges!




\subsection{Structure of the voltage-gated calcium channel Cav1.1 at 3.6 Å resolution}

This paper can be cited with id \texttt{2016-cav-structure}
\cite{2016-cav-structure}.


Cav structure. What's the diference between [2015-cav-structure]


\subsection{The Receptor Site and Mechanism of Action of Sodium Channel Blocker Insecticides}

This paper can be cited with id \texttt{2016-btx-insect}
\cite{2016-btx-insect}.


Homology model to open cockroach channel and docking study




\subsection{Convergent Substitutions in a Sodium Channel Suggest Multiple Origins of Toxin Resistance in Poison Frogs}

This paper can be cited with id \texttt{2016-btx-frogs}
\cite{2016-btx-frogs}.


Evolutionary analysis and docking study on NaV 1.4 in frogs immune to
toxins




\subsection{Structure of the voltage-gated calcium channel Cav1.1 complex}

This paper can be cited with id \texttt{2015-cav-structure}
\cite{2015-cav-structure}.


also a cav structure.


\subsection{Structure of a Prokaryotic Sodium Channel Pore Reveals Essential Gating Elements and an Outer Ion Binding Site Common to Eukaryotic Channels}

This paper can be cited with id \texttt{2014-closed-navae1p}
\cite{2014-closed-navae1p}.


Closed Pore and CTD in pore-only NavAe. Correlates CTD neck unfolding with
activation.




\subsection{Role of the C-terminal domain in the structure and function of tetrameric sodium channels}

This paper can be cited with id \texttt{2013-open-pore-navms}
\cite{2013-open-pore-navms}.


Pore only open conformation. Supposed to have the CTD but rcsb pdb doesn't
show it. Proposes CTD role in gating.




\subsection{Structure of a bacterial voltage-gated sodium channel pore reveals mechanisms of opening and closing}

This paper can be cited with id \texttt{2012-partially-open-pore-navms}
\cite{2012-partially-open-pore-navms}.


Pore only partially open state. This may or may not collapse to the closed
state in molecular dynamics.




\subsection{Crystal structure of an orthologue of the NaChBac voltage-gated sodium channel}

This paper can be cited with id \texttt{2012-closed-navrh}
\cite{2012-closed-navrh}.


Pore and Voltage Sensing domains in a closed state.




\subsection{Crystal structure of a voltage-gated sodium channel in two potentially inactivated states}

This paper can be cited with id \texttt{2012-closed-asym-navab}
\cite{2012-closed-asym-navab}.


Pore and Voltage Sensing domains in an asymmetric closed state. NavAb.




\subsection{Structural basis for gating charge movement in the voltage sensor of a sodium channel}

This paper can be cited with id \texttt{2012-vsd-mechanism}
\cite{2012-vsd-mechanism}.


Homology modelling to resting, intermediate and activated states of voltage
sensing domains. Disulfide locking experiments. Cites
[doi:10.1073/pnas.0806486105] and [doi:10.1073/pnas.0912307106] and
[doi:10.1073/pnas.1116449108]




\subsection{The crystal structure of a voltage-gated sodium channel}

This paper can be cited with id \texttt{2011-closed-navab}
\cite{2011-closed-navab}.


Pore and Voltage Sensing domains in a closed state. First NaV Structure.
Prokaryote NavAb







\bibliography{nav.bib}

\end{document}