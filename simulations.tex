

% This file is autogenerated. Don't edit it
\documentclass{article}

\usepackage{amsmath}              % need for subequations
\usepackage{amssymb}              % for symbols
\usepackage{hyperref}             % put links on stuff
\usepackage{cleveref}             % use for referencing figures/equations
                                  % hyperref must come before cleverref
\usepackage{natbib}               % biblio
\usepackage{color}                % use if color is used in text
\usepackage{soul}                 % hyphenation, strikethrough
\usepackage{graphicx}
\usepackage{parskip}

\bibliographystyle{unsrt}

\begin{document}

\title{A collection of papers}
\date{\today}

\author{gitbib}

\maketitle

\section{Section title tbd}
\subsection{Markov State Models and tICA Reveal a Nonnative Folding Nucleus in Simulations of NuG2}

This paper can be cited with id \texttt{2016-schwantes-nug2}
\cite{2016-schwantes-nug2}.


They find an intermediate in [2011-larsen-folding] NuG2 trajectories that
is a register shift that was missed before tICA+MSM.




\subsection{Fs MD Trajectories}

This paper can be cited with id \texttt{2014-fs-peptide}
\cite{2014-fs-peptide}.


\subsection{How Fast-Folding Proteins Fold}

This paper can be cited with id \texttt{2011-larsen-folding}
\cite{2011-larsen-folding}.


The authors simulated folding trajectories for 12 small proteins. The
simulations were between 100 us and 1 ms. This paper was a considerable
advance for the field, and more or less closed the book on molecular
dynamics for folding.




\subsection{Atomic-Level Characterization of the Structural Dynamics of Proteins}

This paper can be cited with id \texttt{2010-shaw-fip35-bpti}
\cite{2010-shaw-fip35-bpti}.


Simulation of fip35 ww domain: 2x 100 us. Note this was at 400K so
unfolding could be observed.

Simulation of bpti: 1ms. Note this was done with tip4p for reasons.




\subsection{Transition Networks for the Comprehensive Characterization of Complex Conformational Change in Proteins}

This paper can be cited with id \texttt{2006-noe-conf-change}
\cite{2006-noe-conf-change}.


\subsection{Structural mechanism of the recovery stroke in the Myosin molecular motor}

This paper can be cited with id \texttt{2005-myosin-motor}
\cite{2005-myosin-motor}.





\bibliography{simulations.bib}

\end{document}