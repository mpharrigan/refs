

% This file is autogenerated. Don't edit it
\documentclass{article}

\usepackage{amsmath}              % need for subequations
\usepackage{amssymb}              % for symbols
\usepackage{hyperref}             % put links on stuff
\usepackage{cleveref}             % use for referencing figures/equations
                                  % hyperref must come before cleverref
\usepackage{natbib}               % biblio
\usepackage{color}                % use if color is used in text
\usepackage{soul}                 % hyphenation, strikethrough
\usepackage{graphicx}
\usepackage{parskip}

\bibliographystyle{unsrt}

\begin{document}

\title{A collection of papers}
\date{\today}

\author{gitbib}

\maketitle

\section{Section title tbd}
\subsection{Roads towards fault-tolerant universal quantum computation}

This paper can be cited with id \texttt{2017-fault-tolerant-computation}
\cite{2017-fault-tolerant-computation}.


Superconducting qubits [2013-superconducting-qubit-outlook=3].

Gottesman-Knill theorem says you need T in addition to S, H, and CNOT or
you get no quantum [1997-gottesman-thesis=9] (ref. 10).

Trivial codes can't have transversal implementations of all gates for
universal compuation (ref. 11) (ref. 12).

Surface code first as a topological memory [2002-surface-code=13]. Logical
qubit can be two holes in a code sheet (ref. 17) or two pairs of latice
defects or twists (ref. 18) (ref. 19).




\subsection{Quantum software}

This paper can be cited with id \texttt{2017-insight}
\cite{2017-insight}.


Nature ran a featurette where everyone mused about quantum computing and
how to make it useful. Includes [2017-fault-tolerant-computation],
[2017-quantum-programming-language],




\subsection{Programming languages and compiler design for realistic quantum hardware}

This paper can be cited with id \texttt{2017-quantum-programming-language}
\cite{2017-quantum-programming-language}.


Quantum/classical co-processor model described by [2015-quipper=19].




\subsection{First quantum computers need smart software}

This paper can be cited with id \texttt{2017-rigetti-quantum-software}
\cite{2017-rigetti-quantum-software}.


A comment that argues for good quantum software.




\subsection{Charge- and Flux-Insensitive Tunable Superconducting Qubit}

This paper can be cited with id \texttt{2017-tunable}
\cite{2017-tunable}.


Improve fluxonimum (ref. 10) (ref. 11) (ref. 12) (ref. 13) (ref. 14) (ref.
15) with "sweet spots". I think this is just simulations of how it would
behave w.r.t noise though.

Static qubit-qubit couplings with 2q-gates in hundreds of nanoseconds,
100us coherence, and fidelity of 99.1% (ref. 1) (ref. 2) (ref. 3).

Frequency tinable qubits: 20us coherence, 50ns 2q-gates and 99.44% fidelity
(ref. 4) (ref. 5). Fluctuations from flux noise ruin coherence (ref. 6)
(ref. 7) (ref. 8). Also not anharmonic enough means leaks to higher levels
(ref. 9).




\subsection{Demonstration of Universal Parametric Entangling Gates on a Multi-Qubit
  Lattice}

This paper can be cited with id \texttt{2017-multi-qubit}
\cite{2017-multi-qubit}.


Eight qubits in a ring, alternating fixed and tunable. Do 2q gates.




\subsection{A functional architecture for scalable quantum computing}

This paper can be cited with id \texttt{2016-scalable}
\cite{2016-scalable}.


Quantum simulation algorithsm (ref. 1) (ref. 2) (ref. 3).

Quantum machine learning (ref. 4)

Quantum error correction benchmarks (ref. 5) (ref. 6) (ref. 7).

Variational quantum eigensolvers (ref. 8) (ref. 9) (ref. 10).

Correlated material simulations (ref. 11).

Approximate optimization (ref. 12).

For the problems of catalysts (ref. 13) and high temperature
superconductivity (ref. 9) show promise.

Cryo operation and superconducting materials means no sissipation
preserving quantum coherance.

Transmon qubits have large coherence time (ref. 14). Fluxonium qubits have
wide frequency tunability and strong nonlinearity (ref. 15). This means
fluxonium are better for two-qubit gates.

Quantum limited amplifiers (ref. 16) (ref. 17) (ref. 18): Josephson
parametric amplifier, Josephson bifurcation amplifier, and Josephson
parametric converter. Non-linear resonators.

Can do rotations Rx and Ry on any qubit. Can do SWAP between any transmon
and fluxonium. Can do CPhase between any fluxonium and half the transmons.
All gates can be made with these primitives (ref. 19).

Introduce "TQF" estimate of width * depth of quantum circuit you can run.
(ref. 1) runs electronic structure for very small molecules.

Transmon can be "data" for surface code error correction (ref. 24) (ref.
25) and fluxonium as ancillas for parity measurement.




\subsection{Quantum-Enhanced Machine Learning}

This paper can be cited with id \texttt{2016-quantum-ml}
\cite{2016-quantum-ml}.


\subsection{A Practical Quantum Instruction Set Architecture}

This paper can be cited with id \texttt{2016-quil}
\cite{2016-quil}.


\subsection{Scalable Quantum Simulation of Molecular Energies}

This paper can be cited with id \texttt{2016-h2-vqe}
\cite{2016-h2-vqe}.


Solves molecular hydrogen with variational quantum eigensolver (which is
hybrid quantum - classical) and compares to trotterization and quantum
phase estimation. The VQE is better.




\subsection{Programming the quantum future}

This paper can be cited with id \texttt{2015-quipper}
\cite{2015-quipper}.


Quantum programming language implemented inside Haskell. Invisions quantum
co-processor.




\subsection{Superconducting Circuits for Quantum Information: An Outlook}

This paper can be cited with id \texttt{2013-superconducting-qubit-outlook}
\cite{2013-superconducting-qubit-outlook}.


\subsection{Topological quantum memory}

This paper can be cited with id \texttt{2002-surface-code}
\cite{2002-surface-code}.


Called the seminal work in surface code error correction by
[2017-fault-tolerant-computation], this long article seems to evaluate the
details of the surface code which were introduced in
[1997-kitaev-error-correction=4] and [1997-anyons=5].




\subsection{Fault-tolerant quantum computation by anyons}

This paper can be cited with id \texttt{1997-anyons}
\cite{1997-anyons}.


\subsection{Stabilizer Codes and Quantum Error Correction}

This paper can be cited with id \texttt{1997-gottesman-thesis}
\cite{1997-gottesman-thesis}.


\subsection{Quantum Error Correction with Imperfect Gates}

This paper can be cited with id \texttt{1997-kitaev-error-correction}
\cite{1997-kitaev-error-correction}.


\subsection{Measurements of Macroscopic Quantum Tunneling out of the Zero-Voltage State of a Current-Biased Josephson Junction}

This paper can be cited with id \texttt{1985-macroscopic-quantum-tunneling}
\cite{1985-macroscopic-quantum-tunneling}.





\bibliography{quantum.bib}

\end{document}