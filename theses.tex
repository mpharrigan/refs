

% This file is autogenerated. Don't edit it
\documentclass{article}

\usepackage{amsmath}              % need for subequations
\usepackage{amssymb}              % for symbols
\usepackage{hyperref}             % put links on stuff
\usepackage{cleveref}             % use for referencing figures/equations
                                  % hyperref must come before cleverref
\usepackage{natbib}               % biblio
\usepackage{color}                % use if color is used in text
\usepackage{soul}                 % hyphenation, strikethrough
\usepackage{graphicx}
\usepackage{parskip}

\bibliographystyle{unsrt}

\begin{document}

\title{A collection of papers}
\date{\today}

\author{gitbib}

\maketitle

\section{Section title tbd}
\subsection{Statistical models of protein conformational dynamics}

This paper can be cited with id \texttt{2016-mcgibbon-thesis}
\cite{2016-mcgibbon-thesis}.


Chapter 1 is a bespoke introduction to MD and MSMs

Chapter 2 is adapted from [2013-mcgibbon-kdml=37].

Chapter 3 is adapted from [2014-mcgibbon-hmm=92].

Chapter 4 is adapted from [2015-ratematrix=120].

Chapter 5 is adapted from [2014-mcgibbon-bic=162].

Chapter 6 is adapted from [2015-mcgibbon-gmrq=214].

Chapter 7 is adapted from [2016-sparsetica].

Chapter 8 is adapted from [2015-mdtraj].




\subsection{Automated construction of order parameters for analyzing simulations of protein folding and water dynamics}

This paper can be cited with id \texttt{2015-schwantes-thesis}
\cite{2015-schwantes-thesis}.


Section 1.2 is adapted from [2015-schwantes-ktica=27] and
[2014-mcgibbon-bic=28].

Chapter 2 is adapted from [2014-mcgibbon-bic=28].

Chapter 3 is adapted from [2013-schwantes-tica=73].

Chapter 4 is adapted from [2016-schwantes-nug2=122].

Chapter 5 is adapted from [2015-schwantes-ktica=27].

Chapter 6 is supposed to have been submitted for publication.




\subsection{Inferring protein structure and dynamics from simulation and experiment}

This paper can be cited with id \texttt{2013-beauchamp-thesis}
\cite{2013-beauchamp-thesis}.



\section{Section title tbd}
\subsection{Markov State Models and tICA Reveal a Nonnative Folding Nucleus in Simulations of NuG2}

This paper can be cited with id \texttt{2016-schwantes-nug2}
\cite{2016-schwantes-nug2}.


They find an intermediate in [2011-larsen-folding] NuG2 trajectories that
is a register shift that was missed before tICA+MSM.




\subsection{Efficient maximum likelihood parameterization of continuous-time Markov processes}

This paper can be cited with id \texttt{2015-ratematrix}
\cite{2015-ratematrix}.


\subsection{Variational cross-validation of slow dynamical modes in molecular kinetics}

This paper can be cited with id \texttt{2015-mcgibbon-gmrq}
\cite{2015-mcgibbon-gmrq}.


\subsection{Modeling Molecular Kinetics with tICA and the Kernel Trick}

This paper can be cited with id \texttt{2015-schwantes-ktica}
\cite{2015-schwantes-ktica}.


They introduce kernel tICA as an extension to tICA. This is useful to get
non-linear solutions to the tICA equation. They claim you can estimate
eigenprocesses without building an MSM.

They briefly introduce the transfer operator. They introduce the
variational principle of conformation dynamics per [2011-prinz=25]. They
introduce tICA as maximizing the autocorrelation. They say that solutions
to tICA are the same as solutions to the variational problem per
[2013-noe-tica=28]. Linearity makes them crude solutions.

They explain that a natural approach to introduce non-linearity is to
expand the original representation into a higher dimensional space and do
tICA there. They say this is impractical. The expanded space probably has
to be huge. You can perform analysis in the big representation without
explicitly representing it by using the "kernel trick". They reproduce an
example of the kernel trick from [1998-scholkopf-kernel-pca=39].

They re-write the tICA problem only in terms of inner products so you can
apply the kernel trick. They introduce normalization. They choose a
gaussian kernel. They simulate a four-well potential, muller potential,
alanine dipeptide, and fip35ww. They need to do MLE cross validation over
parameters (kernel width and regularization strength).

This uses so much RAM! Huge matrices to solve (that scale with the amount
of data!!)




\subsection{Perspective: Markov models for long-timescale biomolecular dynamics}

This paper can be cited with id \texttt{2014-msm-perspective}
\cite{2014-msm-perspective}.


Very good perspective on the importance of analysis (particularly MSM
analysis) for understanding large, modern MD datasets. Money quote: "we
believe that quantitative analysis has increasingly become a limiting
factor in the application of MD"


\subsection{Statistical Model Selection for Markov Models of Biomolecular Dynamics}

This paper can be cited with id \texttt{2014-mcgibbon-bic}
\cite{2014-mcgibbon-bic}.


This is before [2015-mcgibbon-gmrq] GRMQ cross-validation. They explicitly
find the volume of voronoi cells (in low number of tIC space) to find a
likelihood. They use AIC/BIC to find the number of states to use. Finding
volumes is tough and you still can't compare across protocols (so you can
basically only scan number of states or clustering method), but! this was
the first paper to seriously suggest using a smaller number of states to
avoid overfitting.




\subsection{Variational Approach to Molecular Kinetics}

This paper can be cited with id \texttt{2014-nuske-variational}
\cite{2014-nuske-variational}.


This paper is largely redundant with [2013-noe-variational=65]. They cite
it as such: "Following the recently introduced variational principle for
metastable stochastic processes,(65) we propose a variational approach to
molecular kinetics."

They perform their variational approach on 2- and 10-alanine in addition to
1D potentials.

This comes after tICA and cites [2013-schwantes-tica=57] and
[2013-noe-tica=58] in the intro, but does nothing further with it. In
particular, they don't note that tICA is just another choice of basis set.

They cite their error paper [2010-msm-error=55].




\subsection{Learning Kinetic Distance Metrics for Markov State Models of Protein Conformational Dynamics}

This paper can be cited with id \texttt{2013-mcgibbon-kdml}
\cite{2013-mcgibbon-kdml}.


Learn scaling of coordinates to better approximate kinetics? Redundant with
tICA.




\subsection{Identification of slow molecular order parameters for Markov model construction}

This paper can be cited with id \texttt{2013-noe-tica}
\cite{2013-noe-tica}.


The Noe group introduces tica concomitantly with [2013-schwantes-tica].
They use the variational approach from [2013-noe-variational] to derive the
tICA equation. They cite a 2001 book about independent component analysis.


\subsection{Improvements in Markov State Model Construction Reveal Many Non-Native Interactions in the Folding of NTL9}

This paper can be cited with id \texttt{2013-schwantes-tica}
\cite{2013-schwantes-tica}.


The Pande group introduces tica concomitantly with [2013-noe-tica]. This
paper uses PCA as inspiration and cites signal processing literature.


\subsection{A Variational Approach to Modeling Slow Processes in Stochastic Dynamical Systems}

This paper can be cited with id \texttt{2013-noe-variational}
\cite{2013-noe-variational}.


I think the point of this versus [2014-nuske-variational] is to be "protein
agnostic". They allude to proteins, but say this is more general. Their
example is a double-well potential.

They introduce the propogator formalism and stipulate that dynamics can be
seperated into "fast" and "slow" components. In contrast to a quantum
mechanics Hamiltonian, we don't know the propogator here. You have to infer
it from data.

They claim the error bound derived in [2010-msm-error=34] is not
constructive, whereas this method *is* constructive.

Math section heavily cites [2010-msm-error=34].

They adapt the Rayleigh variational principle from quantum mechanics, and
cite [1989-szabo-ostlund-qm=43]. They show that the autocorrelation of the
true first dynamical eigenfunction is its eigenvalue, and an estimate of
the first dynamical eigenfunction necessarily has a smaller eigenvalue.
This sets the variational bound. In terms of names that don't seem to be
used now that we're in the future: the Ritz method is for when you have no
overlap integrals (e.g. MSMs) and the Roothan-Hall method is for when you
do (tICA).

They put it to the test on a double well potential. They use indicator
basis functions to make an MSM; hermite basis functions so they still have
no overlap integrals, but smooth functions; and gaussian basis functions
(with overlap integrals). This must have come before tICA because there is
no mention made of it, even though it would fit in nicely.




\subsection{Markov models of molecular kinetics: Generation and validation}

This paper can be cited with id \texttt{2011-prinz}
\cite{2011-prinz}.


Fantastic in-depth intro to MSMs. Figure 1 in this paper is necessary for
understanding eigenvectors. This defines and relates the propogator and
transfer operator. This shows how we compute timescales from eigenvectors.
This discusess state decomposition error and shows that many states are
needed in transition regions.

quote: it is clear that a “sufficiently fine” partitioning will be able to
resolve “sufficient” detail [2010-msm-error].

Cites [2004-nina-msm] for use of the term "MSM".




\subsection{On the Approximation Quality of Markov State Models}

This paper can be cited with id \texttt{2010-msm-error}
\cite{2010-msm-error}.


\subsection{What Is the Relation Between Slow Feature Analysis and Independent Component Analysis?}

This paper can be cited with id \texttt{doi:10.1162/neco.2006.18.10.2495}
\cite{doi:10.1162/neco.2006.18.10.2495}.


\subsection{Transfer Operator Approach to Conformational Dynamics in Biomolecular Systems}

This paper can be cited with id \texttt{2001-schutte-variational}
\cite{2001-schutte-variational}.


Full treatment of transfer operator / propagator and build an MSM for a
small RNA chain.




\subsection{Nonlinear Component Analysis as a Kernel Eigenvalue Problem}

This paper can be cited with id \texttt{1998-scholkopf-kernel-pca}
\cite{1998-scholkopf-kernel-pca}.


\subsection{Separation of a mixture of independent signals using time delayed correlations}

This paper can be cited with id \texttt{doi:10.1103/PhysRevLett.72.3634}
\cite{doi:10.1103/PhysRevLett.72.3634}.


\subsection{Modern Quantum Chemistry: Introduction to Advanced Electronic Structure Theory}

This paper can be cited with id \texttt{1989-szabo-ostlund-qm}
\cite{1989-szabo-ostlund-qm}.


Cited by [2013-noe-variational] for Rayleigh variational method.




\subsection{Understanding Protein Dynamics with L1-Regularized Reversible Hidden Markov Models}

This paper can be cited with id \texttt{2014-mcgibbon-hmm}
\cite{2014-mcgibbon-hmm}.


Use hidden markov models instead of discrete state MSMs.







\bibliography{theses.bib}

\end{document}