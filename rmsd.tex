

% This file is autogenerated. Don't edit it
\documentclass{article}

\usepackage{amsmath}              % need for subequations
\usepackage{amssymb}              % for symbols
\usepackage{hyperref}             % put links on stuff
\usepackage{cleveref}             % use for referencing figures/equations
                                  % hyperref must come before cleverref
\usepackage{natbib}               % biblio
\usepackage{color}                % use if color is used in text
\usepackage{soul}                 % hyphenation, strikethrough
\usepackage{graphicx}
\usepackage{parskip}

\bibliographystyle{unsrt}

\begin{document}

\title{A collection of papers}
\date{\today}

\author{gitbib}

\maketitle

\section{Section title tbd}
\subsection{Fast determination of the optimal rotational matrix for macromolecular superpositions}

This paper can be cited with id \texttt{2009-theobald-rmsd}
\cite{2009-theobald-rmsd}.


This one computes the optimal rotation in addition to (specifically: after)
just computing the minimal RMSD value. It uses [2005-theobald-rmsd=14] for
finding the optimal RMSD (ie leading eigenvalue of key matrix).


\subsection{Rapid calculation of RMSDs using a quaternion-based characteristic polynomial}

This paper can be cited with id \texttt{2005-theobald-rmsd}
\cite{2005-theobald-rmsd}.


Instead of doing matrix diagonalization or inversion, use netwon-raphson
root-finding on a characteristic polynomial.

Mainly builds off of [1987-horn-rmsd=5].




\subsection{Using quaternions to calculate RMSD}

This paper can be cited with id \texttt{2004-dill-rmsd}
\cite{2004-dill-rmsd}.


Similar to [2005-theobald-rmsd], builds off of [1987-horn-rmsd=5]. Proves
identity with normal 3x3 methods.

Derives the derivative of RMSD wrt coordinates, although "it is well known"




\subsection{On the orthogonal transformation used for structural comparisons}

This paper can be cited with id \texttt{1989-kearsley-rmsd}
\cite{1989-kearsley-rmsd}.


[2005-theobald-rmsd] cites for quaternion RMSD


\subsection{A note on the rotational superposition problem}

This paper can be cited with id \texttt{1988-diamond-rmsd}
\cite{1988-diamond-rmsd}.


[2005-theobald-rmsd] cites for quaternion RMSD




\subsection{Closed-form solution of absolute orientation using unit quaternions}

This paper can be cited with id \texttt{1987-horn-rmsd}
\cite{1987-horn-rmsd}.


[2005-theobald-rmsd] cites for quaternion RMSD




\subsection{A solution for the best rotation to relate two sets of vectors}

This paper can be cited with id \texttt{1978-kabsch-rmsd}
\cite{1978-kabsch-rmsd}.


[2005-theobald-rmsd] says this suffers from rotoinversions because it uses
3x3 rotations instead of 4x4 quaternions.




\subsection{A Fast 3 x N Matrix Multiply Routine for Calculation of Protein RMSD}

This paper can be cited with id \texttt{2014-haque-fast-rmsd}
\cite{2014-haque-fast-rmsd}.





\bibliography{rmsd.bib}

\end{document}