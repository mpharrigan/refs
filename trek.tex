

% This file is autogenerated. Don't edit it
\documentclass{article}

\usepackage{amsmath}              % need for subequations
\usepackage{amssymb}              % for symbols
\usepackage{hyperref}             % put links on stuff
\usepackage{cleveref}             % use for referencing figures/equations
                                  % hyperref must come before cleverref
\usepackage{natbib}               % biblio
\usepackage{color}                % use if color is used in text
\usepackage{soul}                 % hyphenation, strikethrough
\usepackage{graphicx}
\usepackage{parskip}

\bibliographystyle{unsrt}

\begin{document}

\title{A collection of papers}
\date{\today}

\author{gitbib}

\maketitle

\section{Section title tbd}
\subsection{Markov modeling reveals novel intracellular modulation of the human TREK-2 selectivity filter}

This paper can be cited with id \texttt{2016-trek2}
\cite{2016-trek2}.


\subsection{Instantaneous ion configurations in the K+ion channel selectivity filter revealed by 2D IR spectroscopy}

This paper can be cited with id \texttt{2016-kratochvil-soft-knock}
\cite{2016-kratochvil-soft-knock}.


\subsection{Polymodal activation of the TREK-2 K2P channel produces structurally distinct open states}

This paper can be cited with id \texttt{mcclenaghan2016polymodal}
\cite{mcclenaghan2016polymodal}.


\subsection{Transmembrane Potential Modeling: Comparison between Methods of Constant Electric Field and Ion Imbalance}

This paper can be cited with id \texttt{2016-melcr-membrane-compare}
\cite{2016-melcr-membrane-compare}.


Compare external electric field with ion imbalance.




\subsection{Allosteric coupling between proximal C-terminus and selectivity filter is facilitated by the movement of transmembrane segment 4 in TREK-2 channel}

This paper can be cited with id \texttt{ren2016allosteric}
\cite{ren2016allosteric}.


\subsection{A Non-canonical Voltage-Sensing Mechanism Controls Gating in K2P K+ Channels}

This paper can be cited with id \texttt{schewe2016non}
\cite{schewe2016non}.


\subsection{K2P channel gating mechanisms revealed by structures of TREK-2 and a complex with Prozac}

This paper can be cited with id \texttt{2015-carpenter-structures}
\cite{2015-carpenter-structures}.


Structures of up and down trek2.

Cites [2010-k2p-review=1] for background.


\subsection{State-independent intracellular access of quaternary ammonium blockers to the pore of TREK-1}

This paper can be cited with id \texttt{rapedius2012state}
\cite{rapedius2012state}.


\subsection{Molecular regulations governing TREK and TRAAK channel functions}

This paper can be cited with id \texttt{noel2011molecular}
\cite{noel2011molecular}.


\subsection{Potassium ions line up}

This paper can be cited with id \texttt{hummer2014potassium}
\cite{hummer2014potassium}.


\subsection{Ion permeation in K+ channels occurs by direct Coulomb knock-on}

This paper can be cited with id \texttt{2014-kopfer-hard-knock}
\cite{2014-kopfer-hard-knock}.


Introduces a new way of simulating a membrane potential: They stack two
membranes on top of one another, creating an "inside" between the two. This
doesn't hurt simulation throughput, because you get twice as much protein
motion data in the same amount of simulation time (ignore extra factor of
log n in system size). This seems to be a refinement on their earlier work
in [2011-kutzner-double-membrane=25].

They hide the startling fact that every time an ion moves through the
channel, they have to instantaneously move it back inside. Benoit has
argued that this instantaneous jump, which can be a 100 mV difference is
rediculous.

The main point of this paper is that ions translocate through the four
sites of potassium channel without any waters between them. This "hard
knock" mechanism is in contrast to a "soft knock" mechanism where the
ion-ion interactions are softened by interviening waters.

They re-refine the xray data to show it is consistent with the hard-knock
mechanism.




\subsection{A hydrophobic barrier deep within the inner pore of the TWIK-1 K2P potassium channel}

This paper can be cited with id \texttt{aryal2014hydrophobic}
\cite{aryal2014hydrophobic}.


\subsection{Constant electric field simulations of the membrane potential illustrated with simple systems}

This paper can be cited with id \texttt{2012-roux-efield}
\cite{2012-roux-efield}.


Constant electric field




\subsection{The pore structure and gating mechanism of K2P channels}

This paper can be cited with id \texttt{piechotta2011pore}
\cite{piechotta2011pore}.


\subsection{Computational Electrophysiology: The Molecular Dynamics of Ion Channel Permeation and Selectivity in Atomistic Detail}

This paper can be cited with id \texttt{2011-kutzner-double-membrane}
\cite{2011-kutzner-double-membrane}.


They introduce the double-membrane scheme for measuring ion conductance.
They do it on a big ol' beta barrel.




\subsection{Multiple modalities converge on a common gate to control K2Pchannel function}

This paper can be cited with id \texttt{bagriantsev2011multiple}
\cite{bagriantsev2011multiple}.


\subsection{Structural mechanism of C-type inactivation in K+ channels}

This paper can be cited with id \texttt{cuello2010structural}
\cite{cuello2010structural}.


\subsection{Domain Reorientation and Rotation of an Intracellular Assembly Regulate Conduction in Kir Potassium Channels}

This paper can be cited with id \texttt{clarke2010domain}
\cite{clarke2010domain}.


\subsection{Molecular Background of Leak K+ Currents: Two-Pore Domain Potassium Channels}

This paper can be cited with id \texttt{2010-k2p-review}
\cite{2010-k2p-review}.


Nice review of K2P two-pore potassium channels. They talk about the wide
variety of regulatory stimuli




\subsection{Principles of conduction and hydrophobic gating in K+ channels}

This paper can be cited with id \texttt{jensen2010principles}
\cite{jensen2010principles}.


\subsection{Two-P-Domain (K2P) Potassium Channels: Leak Conductance Regulators of Excitability}

This paper can be cited with id \texttt{goldstein2001potassium}
\cite{goldstein2001potassium}.


\subsection{The Membrane Potential and its Representation by a Constant Electric Field in Computer Simulations}

This paper can be cited with id \texttt{2008-roux-efield}
\cite{2008-roux-efield}.


Constant electric field




\subsection{Calculating potentials of mean force from steered molecular dynamics simulations}

This paper can be cited with id \texttt{2004-pmf}
\cite{2004-pmf}.


\subsection{Human TREK2, a 2P Domain Mechano-sensitive K+Channel with Multiple Regulations by Polyunsaturated Fatty Acids, Lysophospholipids, and Gs, Gi, and GqProtein-coupled Receptors}

This paper can be cited with id \texttt{lesage2000human}
\cite{lesage2000human}.


\subsection{Molecular Dynamics of the KcsA K+ Channel in a Bilayer Membrane}

This paper can be cited with id \texttt{2000-roux-kcsa-md}
\cite{2000-roux-kcsa-md}.


They run 4ns of MD on KcsA potassium channel.




\subsection{Two types of inactivation in Shaker K+ channels: Effects of alterations in the carboxy-terminal region}

This paper can be cited with id \texttt{hoshi1991two}
\cite{hoshi1991two}.


\subsection{ff14SB: improving the accuracy of protein side chain and backbone parameters from ff99SB}

This paper can be cited with id \texttt{maier2015ff14sb}
\cite{maier2015ff14sb}.


\subsection{UCSF Chimera-a visualization system for exploratory research and analysis}

This paper can be cited with id \texttt{pettersen2004ucsf}
\cite{pettersen2004ucsf}.


\subsection{Amber 14}

This paper can be cited with id \texttt{case2014amber}
\cite{case2014amber}.


\subsection{The potassium permeability of a giant nerve fibre}

This paper can be cited with id \texttt{hodgkin1955potassium}
\cite{hodgkin1955potassium}.





\bibliography{trek.bib}

\end{document}